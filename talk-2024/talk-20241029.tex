% Created 2024-10-29 火 09:59
% Intended LaTeX compiler: pdflatex
\documentclass[dvipdfmx,11pat]{jarticle}
\usepackage[utf8]{inputenc}
\usepackage[T1]{fontenc}
\usepackage{graphicx}
\usepackage{longtable}
\usepackage{wrapfig}
\usepackage{rotating}
\usepackage[normalem]{ulem}
\usepackage{amsmath}
\usepackage{amssymb}
\usepackage{capt-of}
\usepackage{hyperref}
\setlength{\textwidth}{18cm}
\setlength{\oddsidemargin}{-1cm}
\setlength{\evensidemargin}{-1cm}
\setlength{\topmargin}{-2cm}
\setlength{\textheight}{26cm}
\author{鈴木正幸 suzuki@iwate-u.ac.jp (学修支援担当@図書館)}
\date{\today}
\title{  数理情報科学: コンピュータと数式処理 (2) \href{https://github.com/masayuki054/comp\_and\_cal/tree/master/talk-2024}{comp\_and\_cal/talk-2024at master · masayuki054/comp\_and\_cal · GitHub}}
\begin{document}

\maketitle
\#+author masayuki
\section{先週話せなかった事}
\label{sec:org763011d}

\subsection{偏微分の図的理解}
\label{sec:orgc0cf19a}
\href{https://drive.google.com/drive/folders/1lY6qb2Z02iAD\_WdesHNMpmsGecY3ynDa}{図 - Google ドライブ}
\url{https://drive.google.com/file/d/1h5Of22-boX3jjQ8E3QgzzURAhpExSQGw/view?usp=drive\_link}
\subsection{テイラー展開と微分積分学の基本定理}
\label{sec:orgae83a31}
docs/diff-descrete.pdf
\subsection{思考を支える数学}
\label{sec:org6049cfd}
数理のひろがりのマップ \url{maps/数理のひろがり.xmind}
\subsection{python と sympy による数式計算の例}
\label{sec:org482f54b}
\begin{itemize}
\item Google Colabを使った,
\item python プログラミング言語による記号・数式計算
\item 文書作成とプログラミングが同一文書で行なえる (文芸的プログラミング)
\end{itemize}
\href{https://colab.research.google.com/drive/13-xrzx125aD3L4mzB5BhOHAUyLeV6kCX}{python-calc.ipynb - Colab} 
\section{数式処理アルゴリズム}
\label{sec:org534d061}
\subsection{不定積分入門}
\label{sec:orgd2a3d2f}
\href{http://www-sop.inria.fr/cafe/Manuel.Bronstein/publications/issac98.pdf}{symbolic integration tutorial--issac98.pdf}
wikipedia の参考文献にあった
\subsection{規則と簡約化と検索のための計算機代数}
\label{sec:orgf2c0112}

高次多変数代数方程式の解法として別途,資料\url{./gbasis/gbasis.org}を配
布します。

グレブナー基底に関する用語とブッフバーガー算法については,\href{https://ja.wikipedia.org/wiki/\%E3\%82\%B0\%E3\%83\%AC\%E3\%83\%96\%E3\%83\%8A\%E3\%83\%BC\%E5\%9F\%BA\%E5\%BA\%95}{グレブナー基底 - Wikipedia} 
を参照します。

数学と検索と簡約のつながりについて,考えます。人は理論を考え,コン
ピュータに検索してもらいましょう。

多くの変数の高い次数の方程式の解法を,線形代数の概念に翻訳し,線形空間
の概念と計算に帰着します。見通しと効率が良くなります。
\subsection{参考}
\label{sec:org0bc2d19}
\href{https://ja.wikipedia.org/wiki/\%E3\%82\%B0\%E3\%83\%AC\%E3\%83\%96\%E3\%83\%8A\%E3\%83\%BC\%E5\%9F\%BA\%E5\%BA\%95}{グレブナー基底 - Wikipedia} の紹介
\begin{itemize}
\item \href{http://math.shinshu-u.ac.jp/\~hanaki/dvi/gr.pdf}{gr.pdf}
\item 多項式の連立方程式を扱う魔術 -多項式の連立方程式を扱う魔術
\href{https://www.mathsoc.jp/publication/tushin/0303/maruyama3-3.pdf}{maruyama3-3.pdf}
\end{itemize}

不定積分入門                              :algorithm:risch:integral:math:
\begin{itemize}
\item \href{http://www-sop.inria.fr/cafe/Manuel.Bronstein/publications/issac98.pdf}{symbolic integration tutorial--issac98.pdf}
\end{itemize}

微分方程式
\begin{itemize}
\item ./docs/D-加群と計算数学-目次.pdf
\end{itemize}
\section{グレブナー基底と線形代数}
\label{sec:org5b73f93}
\subsection{ガウスの消去法による線形方程式の解法}
\label{sec:org365e63b}

\subsubsection{連立一次方程式と行列表現}
\label{sec:org70e2a81}
連立一次方程式:
\begin{eqnarray}
  x+y+z & = & 3 \\
  x-y-z & = & -1 \\
  x-y+z & = & 1 \\
  \end{eqnarray}

の行列による表現:
\begin{center}
   \begin{array}{rrrr|l}
   x & y & z & c & \\\hline
   1 & 1 & 1 & -3 & x+y+z-3=0 \\
   1 & -1 & -1 & 1 & x -y -z + 1 = 0\\
   1 & -1 & 1 & -1  & x - y +z - 1 =0\\
   \end{array}
\end{center}
\subsubsection{ガウスの消去法}
\label{sec:orgc59ccb0}

簡約: \(x\) を \(-y-z+3\) で置き換える. 順序の高変数を,順序の低い変数か
成る等式で置き換える。

行列操作でいうと,1行目の横ベクトル定数倍して,他の行から引いたベクト
ルで置き換える:

\begin{center}
   \begin{array}{rrrr|l}
   x & y & z & c   \\ \hline
   1 & 1 & 1 & -3    & x+y+z-3=0 \\
   0 & -2 & -2 & 4 & -2y-2z+4=0\\
   0 & -2 & 0 &  2 & -2y+2=0  \\
   \end{array}
\end{center}

2行目を -2で割って,3行目を-2 で割っても解は変らないので,\\

\begin{center}
   \begin{array}{rrrr|l}
   x & y & z & c   \\ \hline
   1 & 1 & 1 & -3    & x+y+z-3=0 \\
   0 & 1 & 1 & -2 & y+z-2=0\\
   0 & 1 & 0 & -1 & y-1=0  \\
   \end{array}
\end{center}

3行目を \(y = -z+2\) で置き換える: 
\begin{center}
   \begin{array}{rrrr|l}
   x & y & z & c & 意味  \\ \hline
   1 & 1 & 1 & -3 & x + y+ z -3 = 0  \\
   0 & 1 & 1 & 2 & y + z - 2 =0 \\
   0 & 0 & -1 & -1 & -z = -1 \\
   \end{array}
\end{center}   

3行目を-1で割って,3角行列ができる:
\begin{center}
   \begin{array}{rrrr|l}
   x & y & z & c &  \\ \hline
   1 & 1 & 1 & -3 & x+y+z=3  \\
   0 & 1 & 1 & -2 & y+z=2\\
   0 & 0 & 1 & -1 & z= 1\\
   \end{array}
\end{center}   

後退消去して,
\begin{center}
\begin{array}{rrrr|l}
   x & y & z & c &  \\ \hline
   1 & 0 & 0 & 1 & x = 1  \\
   0 & 1 & 0 & 1 & y = 1\\
   0 & 0 & 1 & 1 & z= 1\\
   \end{array}
\end{center}   
\subsection{グレブナー基底計算と行列表現とガウスの消去法}
\label{sec:org1446dc4}

\begin{eqnarray}
f1: x^3 - 3 x^2 -y + 1 = 0\\
f2: -x^2 + y^2 - 1 = 0
\end{eqnarray}

の解を求めたい。
\subsubsection{多変数方程式の行列表現}
\label{sec:org2497aaa}
\(f1\) と \(f2\)に含まれる単項式を元とし,元間の順序を決め(ここでは全
順序辞書式),加法に関して,数係数を並べたものを,ベクトルと考える。

多変数方程式系は,下記のような行列で表現できる:
\begin{center}
   \begin{array}{rrrrrr|l}
   x^3 & x^2 & x & y^2 & y  & c & 意味  \\ \hline
   1   & -3  & 0 & 0   & -1 & 1 &  x^3 - 3 x^2 -y + 1 = 0\\
   0   & -1  & 0 & 1   & 0  & -1 & -x^2 + y^2 - 1 =0 \\
   \end{array}
\end{center}
\subsubsection{簡約}
\label{sec:orgd5fc805}

\begin{itemize}
\item \(f2:  x^2 = y^2 - 1\)  の関係を用いて,
\item \(f1:  x^3 = 3 x^2 + y - 1\) の関係を簡約する。
\end{itemize}

簡約に必要な s-多項式 \(f1 - x \times f2\) は,新たな基底ベクトルとなる。

\begin{itemize}
\item \(f1- x \times f2:  (x^3 = 3 x^2 + y -1) - (x^3 = x y^2 - x)\)

\item \(f3: 0 = 3 x^2 - x y^2 + x + y -1\)
\end{itemize}

この操作により,ベクトル空間の基底に,新たに \(x y^2\) が加わることに
なり,ベクトル空間の次元が増えることになる:  

\begin{center}
  \begin{array}{rrrrrrr|l}
  x^3 & x^2 &  x y^2 & x  & y^2 & y  & c & 意味  \\ \hline
  1   & -3  &     0  & 0  & 0   & -1 & 1 & f1: x^3 - 3 x^2 -y + 1 = 0\\
  0   & -1  &     0  & 0  & 1   & 0  & -1& f2: -x^2 + y^2 - 1 =0 \\
  0   & 3   &    -1  & 1  & 0   & 1  & -1& f3: 0 = 3 x^2 - x y^2 + x + y -1\\
  \end{array}
\end{center}

f3の式は,f2でそのまま簡約できる \(f4 = f3+3 f2\)

\begin{itemize}
\item \(f4: (3x^2 -xy^2 + x + y -1) -3(x^2 -y^2+1)\)
\item \(f4: -xy2+x+y2\)

\begin{center}
  \begin{array}{rrrrrrr|l}
  x^3 & x^2 &  x y^2 & x  & y^2 & y  & c & 意味  \\ \hline
  1   & -3  &     0  & 0  & 0   & -1 & 1 & f1: x^3 - 3 x^2 -y + 1 = 0\\
  0   & -1  &     0  & 0  & 1   & 0  & -1& f2: -x^2 + y^2 - 1 =0 \\
  0   & 3   &    -1  & 1  & 0   & 1  & -1& f3: 0 = 3 x^2 - x y^2 + x + y -1\\
 
  \end{array}
\end{center}
\end{itemize}

この操作を続けていくと,下記の行列表現が得られる。 

\begin{center}
  \begin{array}{rrrrrrrrr|l}
  x^2 &  xy &  x  & y^5 & y^4 & y^3  & y^2 & y & c   & 意味  \\ \hline
  1   &   0 &  0  & 0   & 0   & 0    & -1  & 0 &  1  & f2: x^2 - y^2+1 =0 \\
  0   &   1 &  -1 & 0   & -1  & 0    & 11  & 3 & -13 & f7: xy - x -y^4 +11y^2 +3y -13\\
  0   &   0 &  0  & 1   & 1   & -11  & -17 & 9 & 17  & f8: y^5 + y^4 -11y^3 -17y^2+9y+17\\
    \end{array}
\end{center}
\subsection{数式処理システム sage によるGroebner基底計算と固有値法による方程式の求解}
\label{sec:org2abd24e}

有理数(QQ) を係数とする多変数(\(x_1, x_2\)) の
多項式環 \(QQ[x_1, x_2]\) を生成し,
不定元\(x_1\), \(x_2\)を取り出す:

\begin{verbatim}
x_1, x_2 = QQ['x_1,x_2'].gens()
\end{verbatim}

\(x_1\), \(x_2\), \(x_3\) を使って多項式を定義する:
\begin{verbatim}
f_1 = 2 * x_1^3 * x_2 +6*x_1^3 - 2* x_1^2 - x_1* x_2 - 3* x_1 - x_2 +3
f_2 = x_1^3*x_2 + 3*x_1^3 + x_1^2*x_2 + 2*x_1^2
f_3 = 3*x_1^2*x_2 + 9*x_1^2 + 2*x_1*x_2 + 5*x_1 + x_2 -3
f_1
f_2
f_3
\end{verbatim}

\phantomsection
\label{orgb70e1fb}
\begin{verbatim}
2*x_1^3*x_2 + 6*x_1^3 - 2*x_1^2 - x_1*x_2 - 3*x_1 - x_2 + 3
x_1^3*x_2 + 3*x_1^3 + x_1^2*x_2 + 2*x_1^2
3*x_1^2*x_2 + 9*x_1^2 + 2*x_1*x_2 + 5*x_1 + x_2 - 3
\end{verbatim}




\(f_1\), \(f_2\), \(f_3\) を基底とするイデアル \(I\) を生成する:

\begin{verbatim}
I = ideal(f_1, f_2, f_3)
I
\end{verbatim}

\phantomsection
\label{org89f8953}
\begin{verbatim}
Ideal (2*x_1^3*x_2 + 6*x_1^3 - 2*x_1^2 - x_1*x_2 - 3*x_1 - x_2 + 3,
\end{verbatim}

x\_1\^{}3*x\_2 + 3*x\_1\^{}3 + x\_1\^{}2*x\_2 + 2*x\_1\^{}2,
3*x\_1\^{}2*x\_2 + 9*x\_1\^{}2 + 2*x\_1*x\_2 + 5*x\_1 + x\_2 - 3)
of Multivariate Polynomial Ring in x\_1, x\_2 over Rational Field

\(I\)  のグレブナー基底を求める:

\begin{verbatim}
B = I.groebner_basis(); B
\end{verbatim}

\phantomsection
\label{orge3872c1}
\begin{verbatim}
[x_1^2 - 3/2*x_1 + x_2 - 3, x_1*x_2 + x_1 - x_2 + 3, x_2^2 - 4*x_1 - 5/2*x_2 - 3/2]
\end{verbatim}


\(QQ[x_1, x_2]/I\) は,\(1, x_2, x_1\) が基底となることがわかる。

\begin{verbatim}
type(B)
\end{verbatim}

\$QQ[x\_1, x\_2]/I\$における,かけ算表を作成する:

\begin{verbatim}
bases = [1,x_1,x_2]
x_1p = [I.reduce(x_1*a) for a in bases]; x_1p
x_2p = [I.reduce(x_2*a) for a in bases]; x_2p

\end{verbatim}

\phantomsection
\label{org95b0277}
\begin{verbatim}
[x_1, 3/2*x_1 - x_2 + 3, -x_1 + x_2 - 3]
[x_2, -x_1 + x_2 - 3, 4*x_1 + 5/2*x_2 + 3/2]
\end{verbatim}


\begin{verbatim}
P1 = [ f.coefficient(b) for b in [{x_1:0, x_2:0}, {x_1:1, x_2:0}, {x_1:0, x_2:1}] for f in x_1p]; P1
P2 = [ f.coefficient(b) for b in [{x_1:0, x_2:0}, {x_1:1, x_2:0}, {x_1:0, x_2:1}] for f in x_2p]; P2
\end{verbatim}


\phantomsection
\label{org2f25d3a}
\begin{verbatim}
[0, 3, -3, 1, 3/2, -1, 0, -1, 1]
[0, -3, 3/2, 0, -1, 4, 1, 1, 5/2]
\end{verbatim}


\begin{verbatim}
M1 = matrix(QQ,3,3,P1); M1
M2 = matrix(QQ,3,3,P2); M2

\end{verbatim}

\phantomsection
\label{org07ffe13}
\begin{verbatim}

[  0   3  -3]
[  1 3/2  -1]
[  0  -1   1]

[  0  -3 3/2]
[  0  -1   4]
[  1   1 5/2]
\end{verbatim}


\begin{verbatim}
s1 = M1.eigenvalues(); s1
s2 = M2.eigenvalues(); s2

\end{verbatim}

\phantomsection
\label{org8804bd5}
\begin{verbatim}
[0, -0.765564437074638?, 3.265564437074638?]
[3, -2.765564437074638?, 1.265564437074638?]
\end{verbatim}


\begin{verbatim}
[[i,j,yf_1.subs({x_1:s1[i], x_2:s2[j]}), f_2.subs({x_1:s1[i], x_2:s2[j]})] 
  for i in range(3) for j in range(3)]
\end{verbatim}

\phantomsection
\label{org1a93966}
\begin{verbatim}

[[0, 0, 0, 0],
 [0, 1, 5.765564437074638?, 0],
 [0, 2, 1.734435562925363?, 0],
 [1, 0, -1.963057085015917?, 0.2383115901485682?],
 [1, 1, 4.562484917954546?, -0.5538774415303320?],
 [1, 2, 0.?e-17, 0.?e-17],
 [2, 0, 376.9630570850159?, 262.2616884098514?],
 [2, 1, 0.?e-15, 0.?e-16],
 [2, 2, 263.5625150820455?, 183.3663774415304?]]
\end{verbatim}
\end{document}
