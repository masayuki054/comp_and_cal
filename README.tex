% Created 2019-11-04 月 16:54
% Intended LaTeX compiler: pdflatex
\documentclass[a4j]{jarticle}
\usepackage[utf8]{inputenc}
\usepackage[T1]{fontenc}
\usepackage{graphicx}
\usepackage{grffile}
\usepackage{longtable}
\usepackage{wrapfig}
\usepackage{rotating}
\usepackage[normalem]{ulem}
\usepackage{amsmath}
\usepackage{textcomp}
\usepackage{amssymb}
\usepackage{capt-of}
\usepackage{hyperref}
\author{Masayuki Suzuki}
\date{\today}
\title{}
\begin{document}

\tableofcontents

\section{数理情報科学特論 コンピュータと数式処理}
\label{sec:org04a6170}

\subsection{今日の話の流れ}
\label{sec:org6c2aedd}

コンピュータとインターネットと上手に付き合って,
知り,考え,記憶し,思い出しましょう。

\begin{itemize}
\item \href{./org/digital\_tools.org}{思考とメモと文書のためのデジタル・ツール}
\begin{itemize}
\item 知識は構造
\item アウトライナー
\item マインドマップ
\item 文芸的プログラミング
\end{itemize}

\item \href{./org/web.org}{Web進化論}
\begin{itemize}
\item 集合知
\item 知識の構造化
\end{itemize}

\item \href{./org/comp\_thinking.org}{計算論的思考}
\begin{itemize}
\item computer科学者のように考え,問題に取り組み,システムをデザイン
しよう
\end{itemize}

\item \href{./org/math-soft.org}{数学ソフトウェア}
\begin{itemize}
\item フリーソフトウェア
\item 数式処理システムと計算機代数アルゴリズム
\begin{itemize}
\item 数値計算を補う
\end{itemize}
\end{itemize}
\end{itemize}

数学と検索と簡約のつながりについて,考えます。
人は理論を考え,コンピュータに検索してもらいましょう。

多くの変数の高い次数の方程式の解法を,
線形代数の概念に翻訳し,
線形空間の

\begin{itemize}
\item 計算機代数アルゴリズムの紹介
\begin{itemize}
\item \href{./org/groebner.org}{グレブナー基底}
\end{itemize}
\end{itemize}

\subsection{\href{./org/links.org}{関連リンク}}
\label{sec:orgb9d1a57}
\end{document}
