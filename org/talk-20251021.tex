% Created 2025-10-20 月 16:13
% Intended LaTeX compiler: pdflatex
\documentclass[dvipdfmx,11pat]{jarticle}
\usepackage[utf8]{inputenc}
\usepackage[T1]{fontenc}
\usepackage{graphicx}
\usepackage{longtable}
\usepackage{wrapfig}
\usepackage{rotating}
\usepackage[normalem]{ulem}
\usepackage{amsmath}
\usepackage{amssymb}
\usepackage{capt-of}
\usepackage{hyperref}
\setlength{\textwidth}{18cm}
\setlength{\oddsidemargin}{-1cm}
\setlength{\evensidemargin}{-1cm}
\setlength{\topmargin}{-2cm}
\setlength{\textheight}{26cm}
\setcounter{secnumdepth}{6}
\author{鈴木正幸 suzuki@iwate-u.ac.jp (学修支援担当@図書館)}
\date{\today}
\title{ 数理情報科学研究のための数理情報リテラシー}
\begin{document}

\maketitle
\begin{itemize}
\item 講義内容の \href{https://github.com/masayuki054/comp\_and\_cal/}{GitHub リポジトリ} {[}リンク]
\item 講義内容 \href{https://github.com/masayuki054/comp\_and\_cal/blob/master/org/talk-20251021.org}{talk-20251021.org} {[}リンク]
\end{itemize}
\section{あらすじ -- 数理情報科学研究のための数理情報リテラシー}
\label{sec:org5aa60b1}

{[}\href{../Maps/数理情報科学のための情報計算リテラシー.xmind}{Mindmap書類}][\href{shell:xmind ../Maps/数理情報科学のための情報計算リテラシー.xmind}{アプリ}]

\begin{center}
\includegraphics[width=18cm]{../Figs/infocalc-literacy-for-mathinfosci.png}
\end{center}

数理情報リテラシという言葉で,コンピュータやインターネットを用いた,情
報,文書,計算,プログラムなどの理解・作成・保存と共有,多大なデータの
計算や表示など,を行なう際の,考え方とツールの使い方を意味することにし
ます。

情報計算の基盤について概観 (添付の mindmap 図) します。
\begin{itemize}
\item \href{../Maps/ICTの利用目的.xmind}{Mindmap書類}
\item \href{shell:xmind ../Maps/ICTの利用目的.xmind}{Xmindアプリ}]
\end{itemize}

本講では,理工学研究分野での数理情報リテラシとして,
大きく次の点についてお話しします。

\begin{itemize}
\item 数理情報処理リテラシと思考技術

数理情報処理は,結局,デジタル・ツールを用いた思考過程であり思考技術です。
知的な思考技術について紹介し,支援するソフトウェアについて紹介します。
知的思考技術について,集合,関数,関係,グラフ,数理論理学,など分野
からその数理を考えてみます。その数理は,思考の方向を定め,迷ったり,
ゆきづまったりした際の拠り所となると思います。

\item 離散的な計算の原理 (数学的理解とプログラミングのために)

微分を例に,連続関数を離散的に捉え,コンピュータで計算可能にするとと
もに,直観的・図的に計算を表現することを考えます。

\item 数理科学的な計算道具については,計算科学とは何か,どんなプログラムが
あるのかについてお話しします。特に,重要な,偏微分方程式を解くための
有限要素法を例にお話しします。数式処理との関連についても。

\item 数学的計算を支援するソフトウェアについて紹介します。

\item python, Google colab を利用した,微積分計算ノートを紹介します。

\item 計算,検索,探索,数理とはについて考えるために,数式処理における重要
なアルゴリズムである,グレブナー基底計算と多変数代数方程式の解法につ
いて概観します。
\end{itemize}
\section{知的思考の技術}
\label{sec:org16abac1}
\subsection{思考の基本}
\label{sec:org81cd516}
\begin{itemize}
\item 分類 [googling AIモード 思考 分類 抽象化]

\item 抽象化 [googling AIモード 思考 抽象化 帰納]
ある関係性で複数の事柄をまとめ,適切な名前を付けること

\item 詳細化 [googling AIモード 思考 詳細化  具体化]
抽象化の逆。記号の意味を示すこと。
\end{itemize}
\subsection{わかるとは}
\label{sec:orga7ba00e}

{[}googling: 思考 わかるとは 理解]

\href{https://www.chikumashobo.co.jp/product/9784480059390/}{『「わかる」とはどういうことか』山鳥 重 | 筑摩書房 }
の読書ノート参照
\begin{itemize}
\item \href{../Maps/わかるとは.xmind}{わかるとは (file-open)} または
\item \href{shell:xmind ../Maps/わかるとは.xmind}{わかるとは (xmind-open)}
\end{itemize}
\subsection{知的思考の技術の7つのステップ}
\label{sec:org979dfa2}

思考することについて思考するのが知的思考(メタ思考ともいう)。
思考はプロセスで,プロセスをサブプロセスに分解し,名前と意味を付ける。
すると思考プロセスは,思考の対象にすることができる。

\href{https://www.amazon.co.jp/\%E7\%9F\%A5\%E7\%9A\%84\%E6\%80\%9D\%E8\%80\%83\%E3\%81\%AE\%E6\%8A\%80\%E8\%A1\%93\%E2\%80\%95\%E8\%80\%83\%E3\%81\%88\%E3\%82\%8B\%E3\%83\%95\%E3\%83\%AC\%E3\%83\%BC\%E3\%83\%A0\%E3\%82\%92\%E5\%BC\%B7\%E5\%8C\%96\%E3\%81\%99\%E3\%82\%8B7\%E3\%81\%A4\%E3\%81\%AE\%E3\%82\%B9\%E3\%83\%86\%E3\%83\%83\%E3\%83\%97\%E3\%81\%AE\%E6\%80\%9D\%E8\%80\%83\%E8\%A1\%93-SANNO\%E3\%83\%9E\%E3\%83\%8D\%E3\%82\%B8\%E3\%83\%A1\%E3\%83\%B3\%E3\%83\%88\%E3\%82\%B3\%E3\%83\%B3\%E3\%82\%BB\%E3\%83\%97\%E3\%83\%88\%E3\%82\%B7\%E3\%83\%AA\%E3\%83\%BC\%E3\%82\%BA-\%E7\%94\%A3\%E6\%A5\%AD\%E8\%83\%BD\%E7\%8E\%87\%E5\%A4\%A7\%E5\%AD\%A6\%E7\%B7\%8F\%E5\%90\%88\%E7\%A0\%94\%E7\%A9\%B6\%E6\%89\%80\%E7\%9F\%A5\%E7\%9A\%84\%E6\%80\%9D\%E8\%80\%83\%E3\%81\%AE\%E6\%8A\%80\%E8\%A1\%93\%E7\%A0\%94\%E7\%A9\%B6\%E3\%83\%97\%E3\%83\%AD\%E3\%82\%B8\%E3\%82\%A7\%E3\%82\%AF\%E3\%83\%88/dp/4382055784}{Amazon.co.jp: 知的思考の技術―考えるフレームを強化する7つのステップの思考術 (SANNOマネジメントコンセプトシリーズ) : 産業能率大学総合研究所知的思考の技術研究プロジェクト: Japanese Books}
のまとめ mindmap:
\begin{itemize}
\item \href{../Maps/知的思考の7段階.xmind}{知的思考の7段階(file)} または
\item \href{shell:open ../Maps/知的思考の7段階.xmind}{xmind\_open}

\item 目的探索の思考
\begin{itemize}
\item 妥当か
\item 適度か
\item 本質的か
\end{itemize}
\item 観察の思考
\begin{itemize}
\item 関連項目の列挙
\end{itemize}
\item 発想の思考
\begin{itemize}
\item 項目リストから思い付く新たなことの列挙
\end{itemize}
\item 分類の思考
\begin{itemize}
\item 項目間の関係によって分類,グループ化
\end{itemize}
\item 構造化の思考
\begin{itemize}
\item グループ間の関係
\begin{itemize}
\item 依存関係
\item 順序関係
\end{itemize}
\end{itemize}
\item 意志決定の思考
\begin{itemize}
\item 理解は既存の知識に整合するか
\end{itemize}
\item 表現の思考
\begin{itemize}
\item 理解から知識へ,
\item 客観的な外化
\end{itemize}
\end{itemize}
\subsection{知識や理解は構造を持った情報}
\label{sec:orgc5ba946}

理解や知識の構造を読み書けるソフトウェアが必要
\begin{itemize}
\item アウトライナー
\item マインドマップ
\item グラフ的文書作成
\end{itemize}
\subsection{知的作業は構造化されたプロセス}
\label{sec:org14ee516}
知的作業の
\begin{itemize}
\item プロセスを見える化し,
\item 作業の流れを手続き化し,
\item コストを見積り,
\item 作業全体をデザインしよう
\end{itemize}
\subsection{プロセス管理手法}
\label{sec:org76d35e1}
\begin{itemize}
\item GTD Getting Things Done (後述)
\item {[}[\url{./topics/comp\_thinking.org}][計算論的思考]] (後述)

\item 文芸的プログラミング
\begin{itemize}
\item プログラムと文書を統合し作成する (google colab)
\item Emacs org-mode
\end{itemize}
\end{itemize}
\section{研究のための数理情報リテラシ}
\label{sec:org05e7f5a}

思考し実験しプログラムし計算し理解し記録し文書を作成するためのコンピュー
タと情報のリテラシ

\begin{itemize}
\item 基本技術は思考法, 文書作成も思考法,プログラミングも思考法
\item 思考は情報処理プロセス
\item メタ思考が重要 (思考する自分を客観視すること)
\begin{itemize}
\item 思考プロセスの見える化
\item 知的思考の \textbf{技術} がある
\end{itemize}
\end{itemize}
\subsection{考える,記憶する,理解する,こととは?}
\label{sec:orgc48bef1}
\subsubsection{思考の基本 抽象化と詳細化, So What, Why so, MECE (mutually}
\label{sec:org522b1ed}
exclusive collectively exhaustive)
{[}googling AIモード 思考 So What, Why so]
\url{elisp:google-this "AIの思考ステップ")}
\subsubsection{思考の記録 MECE, 構造化,GTD}
\label{sec:org8cb7b2c}
{[}googling 思考の記録 構造化 GTD]
\subsubsection{いろいろな思考法がある}
\label{sec:orgf9b364d}
\begin{itemize}
\item 論理的思考  \url{elisp:(google-this "思考" "論理的思考")} へんな回答がかえって
きた
\url{elisp:(google-this "思考と論理的思考とAI")}

\item 批判的思考 [googling 思考 批判的思考]
\item 発想法 [googling 思考 発想法]
\item 水平思考 [googling 思考 水平思考]
\item 計算論的思考[googling 思考 計算論的思考]
\end{itemize}
\subsubsection{情報リテラシーは思考法}
\label{sec:org4e27797}
\begin{itemize}
\item 生のデータに意味付けして情報になる
\item 知的情報リテラシーの技術がある
\begin{itemize}
\item 情報の収集
\item 情報の加工
\item 情報の分析
\item 情報の蓄積
\item 情報の生成
\end{itemize}
\item 知的思考技術と対応している
\end{itemize}
\subsubsection{理解の記録(外化,メモや文書)}
\label{sec:org9295191}
思考の目的は理解すること
\begin{itemize}
\item わかるとは.xmindの参照
\begin{itemize}
\item \href{../Maps/わかるとは.xmind}{わかるとは.xmind}|\href{shell:xmind  ../Maps/わかるとは.xmind}{xmind-open}
\end{itemize}
\end{itemize}

理解と外化(メモ)  
\begin{itemize}
\item 理解を外化する. 小さな理解のくりかえしを,構造的に記録する
\item 内化しやすい外化
\begin{itemize}
\item 論理的に理解しやすい外化はアウトライン構造
\item イメージしやすい外化は,マインドマップ構造
\end{itemize}
\item 外化の意味付
\begin{itemize}
\item 情報を外に出すことは二つの意味がある
\begin{itemize}
\item 内的情報を外部記憶に置くこと
こっちが外化, これがメモ
\item 自分の情報を他者が理解できる形にすること
第三者への情報 客観化,文書化
\end{itemize}
\end{itemize}
\item 理解と知識と外化
\begin{itemize}
\item 小さな理解と大きな理解,整合性の問題
\item 理解の再帰性
\end{itemize}
\end{itemize}
\subsubsection{コンピュータとインターネットを用いる意味}
\label{sec:org14bc6cf}
思考と記憶を補助し,外化を促すこと
\begin{itemize}
\item 情報を保存し共有すること
\item 検索できるようにすること
\item 人と計算機の共同作業の実現,互いに補完と拡張
\end{itemize}
\subsection{プロセス管理法}
\label{sec:orgc231a15}
\begin{itemize}
\item 計算論的思考 \href{https://masayuki054.github.io/cloud\_and\_information\_literacy/docs/computational\_thinking.html}{クラウドと情報リテラシー 計算論的思考}
\item {[}GTD (Getting Things Done) プロセス管理法] googling 
\begin{itemize}
\item Todo リストの作り方と管理方法
\end{itemize}
\end{itemize}
X  - \href{https://masayuki054.github.io/cloud\_and\_information\_literacy/i\_maps/\%E5\%83\%95\%E3\%81\%AB\%E3\%81\%A8\%E3\%81\%A3\%E3\%81\%A6\%E3\%81\%AEGTD\_imap.pdf}{僕にとってのGTD\_imap.pdf}
\begin{itemize}
\item ストレスフリーなプロセス管理 すべて忘れても思い出せる
\end{itemize}
\begin{itemize}
\item \url{./topics/9つのストレス対処法.org}
\end{itemize}
\section{思考の数学的理解}
\label{sec:org9649964}
\href{https://github.com/masayuki054/mathematics/tree/main}{GitHub - masayuki054/mathematics: 岩手大学講義 数理のひろがりのためのリポジトリ}
\subsection{数学は言葉である}
\label{sec:orgf6a1e4a}
\url{../Maps/数理のひろがり.xmind} | \url{shell:xmind ../Maps/数理のひろがり.xmind}
\begin{itemize}
\item 数学基礎論
\item 集合と構造 情報
\item 集合と関数 手続きと抽象化
\item 集合と関係 分類と構造
\item 数理論理学 文章と論理
\begin{itemize}
\item 命題 (関数,述語,論理式)
\item 推論規則
\item 定義,前提,証明,定理
\end{itemize}
\item グラフ 構造の可視化
\begin{itemize}
\item \href{https://github.com/masayuki054/mathematics/blob/main/think/graph-thinking-2.org}{mathematics/think/graph-thinking-2.org at main · masayuki054/mathematics · GitHub}
\end{itemize}
\end{itemize}
\subsection{思考のためのデジタルツール}
\label{sec:orga186ae9}
考えることをサポートするデジタルツールがある \href{https://masayuki054.github.io/ict\_literacy\_for\_thinking\_and\_memo/}{メモと思考のためのICTの活用}
\paragraph{思考の可視化}
\label{sec:org49cecf2}
\begin{itemize}
\item グラフで考える
\begin{itemize}
\item \href{https://github.com/masayuki054/mathematics/blob/main/think/graph-thinking-2.org}{mathematics/think/graph-thinking-2.org at main · masayuki054/mathematics · GitHub}
\end{itemize}
\item マインドマップ
\begin{itemize}
\item 放射的思考 発想
\item アウトライン構造 論理的繋り
\end{itemize}
\end{itemize}
\section{数学と計算リテラシ}
\label{sec:orgfa6fb20}
\begin{itemize}
\item 計算科学と数学ソフトウェア
\item プログラミングによる計算プロセス管理
\item 計算のための数学
\begin{itemize}
\item 微分から差分へ
\item 方程式の解法,連立一次方程式
\end{itemize}
\end{itemize}
\subsection{連続な数学を離散的に捉える}
\label{sec:org8cbf44c}
\begin{itemize}
\item 意味を捉え,コンピュータで計算するために,有限・可算的な捉え方が必要
\item 計算は代表点一点で行ない,近傍の点は定数または直線近似する
\item 数値も有限精度,記憶も有限個
\end{itemize}
\subsubsection{連続関数の変化を折れ線グラフで考える}
\label{sec:orgb17e59c}

\href{./docs/diff-descrete.pdf}{2024年度の手書きの資料}
\paragraph{連続関数を折れ線グラフと見做す}
\label{sec:org6e4aff4}

\paragraph{1変数関数の変化}
\label{sec:org65143b3}
\begin{itemize}
\item 関数の変化の図的解釈とインタフェース式

\item 微分の意味(図)とインタフェース式

\item 二階微分の図的解釈とインタフェース式

\item 多階微分の図的解釈とインタフェース式

\item 微積分学の基本定理の図的理解

\item 関数の変化とテイラー展開
\begin{itemize}
\item 関数の変化を微分形式で表わし,
\item {[}0,x]の区間で積分するとマクローリン展開が得られる
\end{itemize}
\end{itemize}
\paragraph{2変数関数の変化}
\label{sec:org189f0e3}
\begin{itemize}
\item 二変数関数の微分の意味とインタフェース式
\item 二変数関数の二階微分の意味とインタフェース式
\end{itemize}

合成関数の描画法・微分の意味,逆関数の微分,関係の微分と微分方程式など
\subsection{計算科学とは}
\label{sec:orgdf87a9b}

たとえば,\href{./topics/宇宙がわかる17の方程式.org}{宇宙がわかる17の方程式} で記述される現象を解析する。

\href{https://ja.wikipedia.org/wiki/\%E8\%A8\%88\%E7\%AE\%97\%E7\%A7\%91\%E5\%AD\%A6}{計算科学 - Wikipedia} とは

{[}googling: 計算科学 モデル] 
\begin{itemize}
\item 数学的モデルとその定量的評価により、計算機を用いて問題を解決する
\item 様々な問題の計算機によるシミュレーションやその他の計算手法の適用を指
す。
\end{itemize}

数値解析は,数式ではなく,実際の数を対象とし,物理現象など現実の対象をモデル化した
ものである。

対象領域をモデル化したプログラムやアプリケーションソフトウェアを開発し、
それに様々なパラメータを与えて実行する。
\subsubsection{数値解析}
\label{sec:orgfcd517b}
\begin{itemize}
\item 精度保証付き数値計算
\item 数値線形代数
\item 常微分方程式の数値解法、偏微分方程式の数値解法、数値積分
\end{itemize}
\subsubsection{数値シミュレーション}
\label{sec:orgd3e85a5}
\paragraph{目的}
\label{sec:orgacde37d}
\begin{itemize}
\item 既知の事象を再構築して理解する(例えば、地震、津波などの自然災害)。
\item 既知のシナリオを最適化する(例えば、工学的プロセスや産業プロセス)。
\item 未来または未知の状況を予測する(例えば、気象、原子レベル以下の粒子の振る舞い)。
\item 気象、飛行機の周辺の気流、自動車衝突時の車体の状況、銀河系の星々の動き、爆発物など
\end{itemize}
\paragraph{数値シミュレーションプログラムの実行}
\label{sec:org341d421}
\begin{itemize}
\item コンピュータのメモリ内に論理的メッシュ(網目)を形成し、個々の領域が実世界
のモデルの空間的な一部分を表すようになっている。
\item 気象の場合、ひとつの点が数キロ平方の領域に対応し、その下の地
理状態、風向き、湿度、温度、気圧といったパラメータが与えられる。
\item プログラムはシミュレートする時間間隔に従って、現在の状態を基に次の状
態を計算する。
\item この計算はモデル化された方程式を解くことで行われる。そのような計算を
次々に行っていくのである。
\end{itemize}
\subsubsection{科学的方法}
\label{sec:org38c7c94}
計算科学は科学の第三の形態で、実験/観測と理論の間を補間するもの、とい
う主張もある。
\subsubsection{有限要素法(FEM)[googling: 有限要素法]}
\label{sec:org9fb32d1}
偏微分方程式ソルバ (PDE)の 最も一般的な数値的解法

\begin{itemize}
\item 有限要素法 (FEM)とは、偏微分方程式 (PDE)の定義域 (W)の近似解を
求めるときに使用する数値的解法です。PDEを解くときの最大の難関は、解
を近似的に表す基底関数を作る工程です。基底関数の作り方は数多くありま
すが、どれを使用するかは選択した定式化によって決まります。

\item 線形 / 非線形 / 座屈 / 熱 / 動的 / 疲労解析で使用できます。
\end{itemize}
\subsection{数学ソフトウェア}
\label{sec:org34e68ca}
\subsubsection{MathLibre  \href{https://www.geogebra.org/m/hShSTr6e}{KNOPPIX/Math->MathLibre}}
\label{sec:orga2bde1b}
\begin{itemize}
\item DVD起動Linuxで,
\item オープン・ソースで,
\item フリーな数学ソフトウェアを収録 

\href{https://www.geogebra.org/m/hShSTr6e}{数学ソフトウェア紹介 - GeoGebraBook}
\end{itemize}
\subsubsection{\href{https://ja.wikipedia.org/wiki/\%E6\%95\%B0\%E5\%AD\%A6\%E3\%82\%BD\%E3\%83\%95\%E3\%83\%88\%E3\%82\%A6\%E3\%82\%A7\%E3\%82\%A2}{数学ソフトウェアとは - Wikipedia}}
\label{sec:org4a95286}
\begin{enumerate}
\item ソフトウェア電卓  
\href{http://ja.numberempire.com/}{数の帝国 - 数学ツール} 数の帝国は、強力な数学ツールと数について
の知識のコレクションです。
\item 数式処理システム
\begin{itemize}
\item \href{https://ja.wikipedia.org/wiki/Sage\_(\%E6\%95\%B0\%E5\%BC\%8F\%E5\%87\%A6\%E7\%90\%86\%E3\%82\%B7\%E3\%82\%B9\%E3\%83\%86\%E3\%83\%A0)}{Sage (数式処理システム) - Wikipedia}
\item \href{http://www.wolframalpha.com/}{Wolfram|Alpha: Computational Knowledge Engine}
\end{itemize}
\item 数値解析
\begin{itemize}
\item \href{https://matome.naver.jp/odai/2136163231573327601}{MATLABの代わりに使えるソフトウェアまとめ- NAVER まとめ}
\item \href{http://signalprocessor.blogspot.jp/2016/}{信号処理のお仕事メモ: 2016} (Octave)
\item \href{http://www.inaba-lab.org/wiki/index.php/Octave\%E5\%85\%A5\%E9\%96\%80}{Octave入門 - 東海大学 コンピュータ応用工学科 稲葉研究室Wiki}
\end{itemize}
\end{enumerate}
\subsubsection{数式処理システム \href{http://www.sagemath.org}{Sage} とは}
\label{sec:orge1f529d}
\href{http://www.gregorybard.com/Sage.html}{Gregory V. Bard} 曰く
\begin{itemize}
\item フリーでオープン・ソースで,
\item Maple, Mathematica, Magma, and Matlab に並らぶ,
\item 数学科の学生に最適な
\item 「コンピュータ代数」システム
\end{itemize}
\paragraph{Sage on the Web(1)}
\label{sec:org6b3dbfa}
\begin{itemize}
\item Webで動く,
\item ノートやデスクトップPCへのインストールは必要ない。
\end{itemize}
\subparagraph{クラウド・サーバ}
\label{sec:orga1aaf53}
\href{http://www.cocalc.com/}{CoCalc.com Sage クラウド・サーバ} 
\begin{itemize}
\item 長めの問題向き,
\item プログラムの保存ができる
\item 登録とログインが必要,
\end{itemize}
\subparagraph{セル・サーバ}
\label{sec:orgf33e24f}
\href{http://sagecell.sagemath.org/}{SageMathCell Server}
\begin{itemize}
\item 短かめの問題向き
\item \href{http://www.gregorybard.com/videos/Sage\_part1.swf}{関数,微分,積分,2次元プロット} の例題動画
\item \href{http://www.gregorybard.com/videos/Sage\_part2.swf}{因数分解,3次元プロット}
\end{itemize}
\subparagraph{例題}
\label{sec:org15c0a98}
\begin{verbatim}
x,y = var('x y')
plot3d(sin(x+y), [x,-pi,pi], [y, -pi, pi])
\end{verbatim}

\phantomsection
\label{org5ef95af}
\begin{verbatim}
Launched jmol viewer for Graphics3d Object
\end{verbatim}
\paragraph{Sagemath アプリ}
\label{sec:orge3f8ff9}
数式処理が,スマホで動くなんて,ほんとにビックリ
\begin{itemize}
\item \href{https://itunes.apple.com/jp/app/sage-math/id496492945?mt=8}{Sage Mathを App Store で}
\item \href{https://play.google.com/store/apps/details?id=org.sagemath.droid\&hl=ja}{Sage Math - Google Play の Android アプリ}
\end{itemize}
\paragraph{入門}
\label{sec:org88a7b04}
\begin{itemize}
\item \href{http://doc.sagemath.org/html/ja/tutorial/index.html}{Sageチュートリアル.ja} すぐ後で,実行しながら,説明します
\item \href{http://doc.sagemath.org/pdf/ja/tutorial/tutorial-jp.pdf}{Sage tutorial-jp.pdf}
\item \href{http://www.pwv.co.jp/\%7Etake/TakeWiki/index.php?sage\%2Fsage\%E3\%81\%AE\%E7\%B4\%B9\%E4\%BB\%8B}{sage/sageの紹介 - PukiWiki} たけもとさん
\item \href{http://www.pwv.co.jp/\~take/TakeWiki/index.php?sage\%2F\%E8\%A8\%88\%E7\%AE\%97\%E3\%81\%97\%E3\%81\%A6\%E3\%81\%BF\%E3\%82\%88\%E3\%81\%86}{sage/計算してみよう - PukiWiki}
\item \href{http://www.sagemath.org/help-video.html}{series of videos/screencasts}
\end{itemize}
\paragraph{例題}
\label{sec:org3273c7c}
\begin{itemize}
\item \href{http://doc.sagemath.org/html/en/constructions/index.html}{プログラムの制作例}
\item \href{https://wiki.sagemath.org/interact}{対話的なデモリポジトリ} 
\begin{itemize}
\item interact / calculus / Taylor  を見てみよう
\end{itemize}
\end{itemize}
\paragraph{Sagemath に関する本 (フリー)}
\label{sec:orgfbb32ed}
\begin{itemize}
\item \href{http://www.gregorybard.com/Sage.html}{Sage for undergraduates, free pdf}   このページ内に pdf へのリンクが
\item \href{http://mosullivan.sdsu.edu/Teaching/sdsu-sage-tutorial/index.html}{Welcome to the SDSU Sage Tutorial}
\end{itemize}
\paragraph{Sagemath に関するいろいろなページ}
\label{sec:orgd232a58}
\begin{itemize}
\item \href{http://sk.sagepub.com/reference}{SAGE Knowledge - Reference} reference の検索
\item \href{https://qiita.com/HirofumiYashima/items/6bb5770961a3b7d33118}{Sageに関するリンク集}
\item \href{http://wiki.sagemath.org/quickref}{large collection of quick-reference cards}
\item \url{http://sagemath.org} \url{http://doc.sagemath.org}
\end{itemize}
\paragraph{Octave との連携}
\label{sec:org7ea5322}
\begin{itemize}
\item \href{http://www.pwv.co.jp/\%7Etake/TakeWiki/index.php?sage\%2FSage\%E3\%81\%A7Octave\%E3\%82\%92\%E4\%BD\%BF\%E3\%81\%86}{sage/SageでOctaveを使う - PukiWiki}
\item \href{http://www.inaba-lab.org/wiki/index.php/Octave\%E5\%85\%A5\%E9\%96\%80}{Octave入門 - 東海大学 コンピュータ応用工学科 稲葉研究室Wiki}
\end{itemize}
\paragraph{\LaTeX{}とSage}
\label{sec:org765226f}
\begin{itemize}
\item オンライン\LaTeX{}サービス  \href{https://oku.edu.mie-u.ac.jp/\~okumura/texonweb/}{\TeX{} を使ってみよう}
\item \href{https://mytexpert.osdn.jp/index.php?LaTeX\%A4\%CB\%A4\%E8\%A4\%EB\%CF\%C0\%CA\%B8\%BA\%EE\%C0\%AE\%A4\%CE\%BC\%EA\%B0\%FA\%A4\%AD}{\LaTeX{}による論文作成の手引き - MyTeXpert}
\item \href{http://sage.math.gordon.edu/home/pub/51/}{Using \LaTeX{} in Sage -- Sage}
\end{itemize}
\subsection{python と sympy による数式計算の例}
\label{sec:org6593f4b}

\begin{itemize}
\item Google Colabを使った,
\item python プログラミング言語による記号・数式計算
\item 文書作成とプログラミングが同一文書で行なえる (文芸的プログラミング)
\end{itemize}

\href{https://colab.research.google.com/drive/13-xrzx125aD3L4mzB5BhOHAUyLeV6kCX}{python-calc.ipynb - Colab} 
\subsection{数式処理アルゴリズム}
\label{sec:org8178c97}
\subsubsection{不定積分入門}
\label{sec:org02e5329}
\href{http://www-sop.inria.fr/cafe/Manuel.Bronstein/publications/issac98.pdf}{symbolic integration tutorial--issac98.pdf}
wikipedia の参考文献にあった
\subsubsection{規則と簡約化と検索のための計算機代数}
\label{sec:org3743323}

高次多変数代数方程式の解法として別途,資料\url{./gbasis/gbasis.org}を配
布します。

グレブナー基底に関する用語とブッフバーガー算法については,\href{https://ja.wikipedia.org/wiki/\%E3\%82\%B0\%E3\%83\%AC\%E3\%83\%96\%E3\%83\%8A\%E3\%83\%BC\%E5\%9F\%BA\%E5\%BA\%95}{グレブナー基底 - Wikipedia} 
を参照します。

数学と検索と簡約のつながりについて,考えます。人は理論を考え,コン
ピュータに検索してもらいましょう。

多くの変数の高い次数の方程式の解法を,線形代数の概念に翻訳し,線形空間
の概念と計算に帰着します。見通しと効率が良くなります。
\section{まとめ}
\label{sec:orga1e78c6}

本講では,理工学研究分野での情報計算リテラシとして,
以下の点についてお話ししたつもりです:

\begin{itemize}
\item 思考について紹介し,知的思考の7つのステップ,
わかるとはどういうことか,が重要だと考えます。

\item 思考法について,論理的思考,批判的思考,発想,抽象化と詳細化,
を紹介し,それらを支援してくれる,デジタルツール,マインドマップとア
ウトライナーを紹介しました。

\item 計算科学のためのフリーソフトウェアについて,どんな分野がありどんなア
プリがあるかについて紹介しました。

\item 数値解析において,微分と線形代数の意味を理解することが非常に大事であること

\item 代数的計算を行ってくれる言語とシステムがあること

\item 代数的アルゴリズムの例として,グレブナー基底計算と多変数代数方程式の解法について概観しました。
\begin{itemize}
\item 多項式のイデアルを,頭項,項順序,簡約規則により,構造化することで,
\item イデアルの要素を0に簡約できるグレブナー基底が計算できる
\item その意味付けにおいて線形代数が有効であり,
\item 正規形,順序,置き換え規則の完備化により,無限集合の中の検索を,
有限集合の中での簡約で計算できる
\item ユークリッドの互除法,ガウスの消去法,グレブナー基底による簡約の間
に共通な考え方が存在する
\end{itemize}
\end{itemize}

\begin{figure}[htbp]
\centering
\includegraphics[width=18cm]{../Figs/usingICT.png}
\caption{ICTの利用目的}
\end{figure}

\vspace{3cm}

\begin{figure}[htbp]
\centering
\includegraphics[width=18cm]{../Figs/Web-revolution.png}
\caption{インターネットが起している変革}
\end{figure}

\begin{figure}[htbp]
\centering
\includegraphics[width=18cm]{../Figs/ict-tools-for-thinking-and-memo.png}
\caption{思考とメモと文書のためのデジタル・ツール}
\end{figure}
\end{document}
