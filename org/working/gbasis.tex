% Created 2018-11-11 日 20:45
% Intended LaTeX compiler: pdflatex
\documentclass[a4j]{jarticle}
\usepackage[utf8]{inputenc}
\usepackage[T1]{fontenc}
\usepackage{graphicx}
\usepackage{grffile}
\usepackage{longtable}
\usepackage{wrapfig}
\usepackage{rotating}
\usepackage[normalem]{ulem}
\usepackage{amsmath}
\usepackage{textcomp}
\usepackage{amssymb}
\usepackage{capt-of}
\usepackage{hyperref}
\setcounter{secnumdepth}{6}
\author{suzuki@iwate-u.ac.jp}
\date{\today}
\title{数理情報科学特論 グレブナー基底}
\begin{document}

\maketitle
\tableofcontents

\begin{itemize}
\item 変数が多くて,

\item 次数が高い,

\item 方程式の根を求める(逐次)アルゴリズム
\end{itemize}

コンピュータがたくさんあった場合,

\begin{itemize}
\item 逐次アルゴリズムから,

\item 独立して計算できる部分問題を取り出し,

\item それぞれの計算機で同時に計算する.

\item 部分問題をどのように渡し,結果をどうもらうか

\begin{itemize}
\item 並列化した事による*通信のオーバーヘッド*

\item 通信量と計算量の比 (\textbf{粒度})
\end{itemize}

\item コンピュータ数に対し,どれくらい速くなるのか

\begin{itemize}
\item 問題自体の*並列度*

\item 台数にたいする速度向上は? (\textbf{スケーラビリティ})
\end{itemize}
\end{itemize}

\section{方程式を解くとは?}
\label{sec:org62722c7}

\begin{itemize}
\item 一次方程式, \$a x = b \$

\begin{quote}
$$x = a^{-1} b$$
\end{quote}

\item 連立一次方程式 (系)
$$(a_{11} x_1 + a_{12} x_2 + \ldots = c_1,$$$$\cdots,$$
$$a_{n1} x_1 + a_{n2} x_2 + \ldots = c_n)$$

\begin{quote}


\begin{itemize}
\item 線形代数, ガウスの消去法

\item 一次方程式, \(a x = b\) に帰着させる
\end{itemize}
\end{quote}

\item 一変数方程式, \(a_n x^n + \cdots + a_0 = b\)

\begin{quote}


\begin{itemize}
\item 根の公式, \(x^n = c\) に帰着させる.

\item 帰着できない時, 数値計算 (ニュートン法) で近似的に求める.
\end{itemize}
\end{quote}

\item 多変数(代数)方程式(系)
\((f_1(x, \ldots, z) =0, \cdots, f_n(x, \ldots, z) =0)\)

\begin{quote}


\begin{itemize}
\item 変数消去, 因数分解

\item 必ず解ける方法を知っていますか?
\end{itemize}
\end{quote}
\end{itemize}

\subsection{基底とは}
\label{sec:org6559507}

\paragraph{問題}
\label{sec:org17aa987}

\begin{quote}
天秤秤と, \(a\) グラムの重りと \(b\) グラムの重りが無数にあるとします.
どんな重さが測れるでしょう?

あるいは, \(c\) グラムを測る事ができますか?
\end{quote}

\begin{itemize}
\item この問題は,不定方程式 \(ax+by=c\) を満たす,整数 \(x\), \(y\) を求
める問題となる.

\item この解は,\(a\) と \(b\) の最大公約数を求める問題に帰着さ れます.

\item \(a\) と \(b\) の組合わせで作れる最小の数は,最大公約数であ
り,それの倍数しか\(a\) と \(b\) の組合わせでは作れません.

\item 最大公約数は\(a\) と \(b\) の組合わせでできる数の集合の*基
底*となります.

\item 上の問題は,\(a\) と \(b\) の最大公約数を\(g\) とすると,
\(c\) は \(g\) の倍数でなければ解が存在しないこと,\(a x + b y = g\)
の \(x\) と \(y\) は,Euclid の互除法によって求められます.
\end{itemize}

\paragraph{問題}
\label{sec:org41bf7c3}

\begin{quote}
二つの方程式 \(f_1(x)=0, f_2(x)=0\) の共通根は?
\end{quote}

それぞれの方程式の根を求めて, 共通な根を求めてもいいですが,

\begin{itemize}
\item 上の議論から,二つの式 \(f_1(x), f_2(x)\) の組み合わせで できる,
最も簡単な(次数の低い)式 (基底) を求め, その根を求める.

\item 基底は, \(f_1(x)\) と \(f_2(x)\) の最大公約多項式(\(g(x)\))となり,
$$A(x) f_1(x) + B(x) f_2(x) = g(x),$$ $$\deg(A(x)) < \deg(f_2(x)),$$
$$\deg(B(x)) < \deg(f_1(x))$$
\end{itemize}

\subsection{多変数方程式 をどう解くか?}
\label{sec:org50d01d1}

$$(f_1(x, \ldots, z) =0, \cdots, f_n(x, \ldots, z) =0)$$

に対し,\(f_1, \ldots, f_n\) を組合わせでできる任意の多項式

$$A_1(x, \ldots, z) f_1(x, \ldots, z) + \cdots + A_n(x, \ldots, z)
f_n(x, \ldots, z)$$

の集合を考えます.

この集合を \((f_1, ... , f_n)\) と表し,\(f_1\) から \(f_n\) が作る
\textbf{イデアル} \({\cal I}\) と呼ぶ.

\({\cal I}\) を作ることのできる多項式の組をイデアルの基底と呼びます.

方程式を解くのに都合の良い基底を求めることは,同じ根を持つ,より簡単な
方程式系への変換となる.

この基底が例えば,

$$(g_1(x, z) =0, g_2(y,z) = 0, \cdots, g_m(z) =0)$$

という形で求まれば,多変数方程式の問題は, 一変数方程式の
問題に帰着される.

「このような変形はできるのか」,「変形する方針は」,「必ず求まるのか」
などが問題となる.

\section{パズルと基底}
\label{sec:org9cc246b}

\paragraph{グラス置き換えパズル}
\label{sec:org51760fa}

ウィスキーのグラス \(W\), ビールのグラス \(B\), お酒のグラス \(S\) が
一列に並んでいる.

グラスは次の置き換え規則で, 置き換えて良いとする.

$$置き換え規則 G \left\{ \begin{array}{rll}
B &  \leftarrow\rightarrow  &  W B\\
BS& \leftarrow\rightarrow   & W  \\
      \end{array}
\right.$$

\paragraph{問題}
\label{sec:org2519068}

\begin{enumerate}
\item \(BSBS\) は \(WWWB\) に置き換えできるか?

\item \(BSBBS\) は \(BWW\) に置き換えできるか?
\end{enumerate}

\paragraph{問題の難しい点}
\label{sec:orgb2426f7}

\begin{itemize}
\item できる場合はその置き換えを示せば良いが,

\item できない事を示す事.
\end{itemize}

\paragraph{パズル解法への道}
\label{sec:org7eb8929}

\begin{itemize}
\item 簡単な方へ置き換える (簡約化)ことにする.
$$簡約規則 R \left\{ \begin{array}{rll}
   WB & \rightarrow  &  B\\
   BS & \rightarrow   & W  \\
         \end{array}
   \right.$$

\item これ以上簡約できないもの (正規形)

\item 置き換え規則 \(G\) で置き換え可能な列の要素は 簡約規則 \(R\)
で同じ正規系を持つか?

この性質が成り立てば, 簡約系で正規形が同じであれば,
置き換え系で,置き換え可能となる.

\item 置き換え可能なのに, 同じ正規形を持たない場合は,
そのような簡約規則を追加すればよい.

例えば, \(WBS\) は二つの

$$\left\{ \begin{array}{rllll}
   WBS & \rightarrow  &  WW\\
   WBS & \rightarrow   & BS & \rightarrow  & W\\
         \end{array}
   \right.$$

置き換え系では, \(WW\) と \(W\) は, \(WBS\) を通して置き換え可能である
から, 簡約系で

$$WW  \rightarrow   W$$

を新しい簡約規則として採用すればいい事になる.

この追加される簡約規則を同やって見付けるかが問題となる.

\item 簡約規則の左項中で, 重なりが生ずるような二つの規則を探す.
(この二つの簡約規則を*危険対*と呼ぶ).

今の場合, \(BS\) と \(WB\) は 重なりを持つ項, \(WBS\) を別の正規形に簡
約する可能性を持つ.

\item この操作を次々に繰り返し, 危険対が全て同じ簡約形を持つよう
になった時, 置き換え可能である物は, 全て同じ正規形を持つ事になる.
\end{itemize}

簡約系の*完備化*という.完備な系とは,

\begin{itemize}
\item 正規系は有限ステップで求まる. (\textbf{停止性})

\item ある項の正規系は, 簡約順序によらず同じになる.(\textbf{合流性})
\end{itemize}

\paragraph{パズルの答え}
\label{sec:org32d9f9d}

簡約規則 \(R\) を完備化すると,

$$簡約規則 R' \left\{ \begin{array}{rll}
WB & \rightarrow  &  B\\
BS & \rightarrow   & W  \\
WW & \rightarrow   & W  \\
      \end{array}
\right.$$

が得られる. これで, \(BSBS \rightarrow^* W\), \(WWWB \rightarrow^* B\),
なので,置き換え可能ではない.

\(BSBBS \rightarrow^* BW\), \(BWWW \rightarrow^* BW\),
なので,置き換え可能となる.

*これがどう方程式と関係しているのでしょう? *

\section{グレブナー基底}
\label{sec:org073a21e}

与えられた方程式\(f_i\) の最高順位項を \(head(f_i)\) 、残りの項を
\(rest(f_i)\) とすると, $$f_i = head(g_i) + rest(g_i)  = 0$$ から
$$head(g_i) \rightarrow - rest(g_i)$$ という簡約規則を作る事ができる.

このような簡約系を作るには, 項間の順序, 簡約, 危険対の求め方を,
方程式用に決める必要がある.

\subsection{項の間の順序}
\label{sec:org5f3ddbf}

いくつの順序が考えられ, 順序によって完備な簡約系が異る.

\begin{description}
\item[{辞書式順序:}] : \(xyz > yz^3 > z^5\)

\item[{全次数辞書式順序:}] \(x^5 > x^4y > x^3yz\)
\end{description}

\subsection{簡約}
\label{sec:org1d26261}

基底の先頭項を残りの項で置き換える簡約規則と見て,
項をより低順位項で置き換える操作.

\begin{itemize}
\item 例2.1:: \(g_1\) を \(g_2\) でM簡約

\(g_1 = x^4yz - xyz^2~~~(~head(g_1) = x^4yz~~,~~rest(g_1) = xyz^2~)~\)

\(g_2 = x^3yz - xz^2~~~(~head(g_2) = x^3yz~~,~~rest(g_2) = xz^2)~\)
\end{itemize}

$$\begin{array}{ll}
g' & = g_1 - ( head(g_1) / head(g_2) ) g_2 \\
   & = g_1 - ( x^4yz / x^3yz ) g_2 \\
   & = x^2z^2 - xyz^2
\end{array}$$

\subsection{S多項式}
\label{sec:orgd737c89}

新たな簡約規則を得るための計算.

2つの多項式 \(f_1,f_2\) のS多項式を \(Sp(f_1,f_2)\) と書き、以下のように計算する。

$$Sp(f_1,f_2)= \frac{lcm}{head(f_1)}f_1 - \frac{lcm}{head(f_2)}f_2$$

\begin{itemize}
\item 例2.2:: \(g_1\) と \(g_2\) のS多項式

\(g_1 = x^3yz - xz^2,\ \   head(g_1) = x^3yz\)

\(g_2 = x^2y^2 - z^2,\ \  \  head(g_2) = x^2y^2\)

\(lcm(head(g_1), head(g_2)) = x^3y^2z\)

$$\begin{array}{ll} 
  Sp(g_1,g_2) &= ( lcm / head(g_1) ) g_1 - ( lcm / head(g_1)) g_2 \\
            &= ( x^3y^2z / x^3yz ) g_1 - ( x^3y^2z / x^2y^2 ) g_2 \\
            &= -xyz^2 + xz^3
  \end{array}$$
\end{itemize}

\subsection{グレブナー基底の定義}
\label{sec:orgb27a5db}

イデアル \({\cal I}\) の基底を \(G={f_1,\cdots,f_n}\) とする。

\(F\) を可能な限りM簡約した結果を \(F'\) とし,

$$F \stackrel{G}{\longmapsto} F'$$ と表す.

\({\cal I}\) の任意の要素 \$f\$に対し,

$$f \stackrel{G}{\longmapsto} 0$$
という性質を持つとき,\(G\) をグレブナー基底と呼ぶ。

\(G\) がグレブナー基底の時,\(f \stackrel{\psi}{\longmapsto} f'\) を計算
し,\(f'=0\) を調べることで、\(f \in {\cal I}\) であるかを簡単に決定できる.

\begin{description}
\item[{例2.3}] \(f_1,f_2,f_3\) のグレブナー基底を求める。(全次数辞書式順序)*
\end{description}

\$\$\left$\backslash${
\begin{array}{l}
f_1 = 2{x_1}^3x_2 +6{x_1}^3-2{x_1}^2-x_1x_2-3x_1-x_2+3\vspace{.2in}\\
f_2 = {x_1}^3x_2 + 3{x_1}^3 + {x_1}^2x_2 + 2{x_1}^2\vspace{.2in}\\
f_3 = 3{x_1}^2x_2 + 9{x_1}^2 + 2x_1x_2 + 5x_1 + x_2 -3
\end{array}
\right.\$\$

(s多項式の例)

\(\begin{array}{ll} 
Sp(f_1,f_2) &= ( lcm / head(f_1) ) f_1 - ( lcm / head(f_1)) f_2 \\
            &= ( 2{x_1}^3x_2 / 2{x_1}^3x_2 ) f_1 - ( 2{x_1}^3x_2 / {x_1}^3x_2 ) f_2 \\
        &= -2x_1^2 x_2 -6x_1^2 -x_1 x_2 - 3x_1 -x_2 +3 = f'_4
\end{array}\)

(M簡約の例)

\(\begin{array}{lll}
f'_4 & \stackrel{f_3}{\longmapsto} & f'_4 - (-2x_1^2 x_2 /{head(f_3)})f_3 \\
     & = & x_1 x_2+ x_1 -x_2 +3
\end{array}\)

<\(f_1,f_2,f_3\) のグレブナー基底>
$$G = [x_1 x_2 + x_1- x_2 + 3, 2{x_1}^2 - 3 x_1 + 2 x_2 - 6, 
    2{x_2}^2 - 8 x_1 - 5 x_2 -3]$$

\section{グレブナー基底から方程式の根を求める方法}
\label{sec:org12ab7ed}

辞書式順序で基底計算を行うと、連立方程式の解が求めやすいが、
基底計算に時間がかかる上に計算量が多くなる.

簡単に求まる基底から,根を求める手法として固有値法がある.

\begin{enumerate}
\item 任意の多項式を, グレブナー基底 \(G\) で簡約した多項式の集合
\({\cal P}^s /{\cal I}\) は, ベクトル空間をなす.

\item グレブナー基底の最高順位項で割り切れない全ての項の集合を Normal
setといい、 \({\cal P}^s /{\cal I}\) ベクトル空間の基底となる。

\item Normal set により \(x_i \times\) を行列で表す事ができる.

\item その行列の固有値は, \({\cal I}\) の \(x_i\) に関する根となる.
\end{enumerate}

\textbf{例3.1:} 例2.3の \(f_1,f_2,f_3\) の根を求める。

<\(f_1,f_2,f_3\) のグレブナー基底>
$$G = [x_1 x_2 + x_1 - x_2 + 3, 2{x_1}^2-3x_1+2x_2-6,2{x_2}^2-8x_1-5x_2-3]$$

$$Normal \; Set =  \{1,x_2,x_1\}$$

<書き換え規則>

\$\$\left\{
\begin{array}{rl}
x_1x_2 & \rightarrow  -x_1+x_2-3\\
{x_1}^2 & \rightarrow  \frac{3}{2}x_1-x_2+3\\
{x_2}^2 & \rightarrow  4x_1+\frac{5}{2}x_2+\frac{3}{2}
\end{array}
\right.\$\$

\(P=c_1\vec{x_1} + c_2\vec{x_2} + c_3\)

<\(x_1 \times\) の行列>

$$\begin{array}{c|ccc}
    &  1 & x_2 & x_1\cr
    \hline
    1   &  0 &  0 &  1 \cr
    x_2 & -3 &  1 & -1 \cr
    x_1 &  3 & -1 & 3/2 \cr
\end{array}$$

< \(x_2 \times\) の行列>

$$\begin{array}{c|c c c}
    &  1 & x_2 & x_1\cr
    \hline
    1   &  0  &  1  &  0 \cr
    x_2 & 3/2 & 5/2 &  4 \cr
    x_1 & -3  &  1  & -1 \cr
\end{array}$$

<\(x_1\) の固有値>

$$\left[ 0, \; \frac{5}{4}+\frac{1}{4}\sqrt{65},
\; \frac{5}{4}-\frac{1}{4}\sqrt{65}\right]$$

<\(x_2\) の固有値>

$$\left[ 3, \; -\frac{3}{4}+\frac{1}{4}\sqrt{65},
\; -\frac{3}{4}-\frac{1}{4}\sqrt{65}\right]$$

これらの固有値が \(f_1,f_2,f_3\) の根である。

\section{Buchberger算法と並列化}
\label{sec:orgd5a6fbc}

以下に,\(f_1, \ldots, f_l\) が作るイデアルの Gröbner基底を計算する
Buchberger算法を示す.

\paragraph{Buchberger算法}
\label{sec:orgc515b07}

\begin{quote}
Input: \$F = $\backslash${ f\_1, \ldots, f\_l$\backslash$}\$\\
Output: Gröbner基底 \(G\) of \(Ideal(F)\)

\#+BEGIN\_QUOTE
  \(PairQ \longleftarrow \phi\);\\
  \(G \longleftarrow \phi\);

\emph{foreach} (\(f_i \in F\)) \(\{\)

\#+BEGIN\_QUOTE
  \$ PairQ \longleftarrow UpdatePairQ(PairQ, f\_i, F)\$;\\
  \$ G \longleftarrow UpdateBase(G, f\_i)\$;
\end{quote}

\(\}\)

\emph{while} ( \(PairQ \ne \phi\) ) \(\{\)

\begin{quote}
\((g_i, g_j) \longleftarrow\) \emph{select an element of \(PairQ\)};\\
\(PairQ \longleftarrow PairQ \setminus \{(g_i, g_j)\}\);\\
\(g_k \longleftarrow {{{\mathrm{SPOL}}(g_i, g_j)\!\downarrow_{G}}}\);

\emph{if} \$g\_k \(\ne\) 0 \$ \(\{\)

\#+BEGIN\_QUOTE
  \(PairQ \longleftarrow UpdateQ(PairQ, g_k, G)\);\\
  \(G \longleftarrow UpdateBase(G, g_k)\);
\end{quote}

  \(\}\)
\#+END\_QUOTE

    \(\}\)
  \#+END\_QUOTE
\#+END\_QUOTE

\paragraph{算法の概要と戦術}
\label{sec:orgb1434b2}

\begin{itemize}
\item \$G\$は中間的な基底の集合,\(PairQ\) は新たな基底を構成可能な
中間基底の組 (\textbf{ペア})の集合,を表している.

\item \(PairQ\) から一つのペア \((g_i, g_j)\) を選ぶ.
この選び方を選択戦術と呼ぶ.

\item \({\mathrm{SPOL}}(g_i, g_j)\) の現在の中間基底での正規形\$g\_k\$を求める.
簡約基底の選び方の順序や簡約法を簡約化戦術と呼ぶ.

\item \$g\_k\$が0でなければ,

\begin{itemize}
\item ペア削除戦術により,新たなペアの生成と,不必要なペアの削除をお
こない(\(UpdateQ\)),

\item 中間基底に追加し,基底削除戦術により不必要な中間基底の削除をおこ
なう (\(UpdateBase\)),
\end{itemize}

\item \(PairQ\) が空になった時点で算法は停止し, \(G\) に Gröbner
基底が求まる.
\end{itemize}

\subsection{Buchberger算法の並列性}
\label{sec:orgf541f36}

Buchberger算法の計算上の問題点は,ペアの個数の組み合わせ的な膨張と,中
間基底の数係数の膨張である.ペアの個数の膨張を防ぐために,いくつかの選
択戦術が考えられており,選択戦術を保持したまま,ペアの個数に関する並列
性の導入が必要となる \cite{strategy-accurate}.

野呂ら\cite{noro97-ap}は, 数係数の膨張による計算時間の増大を,並列計算
により減らせることを示した.筆者\cite{asir-para}は,共有メモリを用いて
更に高速化を行った.一つの基底によるS多項式の簡約
(\({{{\mathrm{SPOL}}(g_i, g_j)\!\downarrow_{\{g_k\}}}}\))
を\$\{\mathrm{SPOL}\}(g\_i, g\_j)\$や\$g\_k\$を分割
し,並列計算する.これを*一簡約並列*と呼ぶ.この方式では,

\begin{itemize}
\item 全ての戦術を保持したまま並列計算が可能であるが,

\item 細粒度の並列化であり,有効となるのは数係数が大きくなった場合に限る,

\item 逐次部分が残る.
\end{itemize}

この方式は,大規模なGröbner基底計算\cite{noro97-mckay}において,並
列度が中規模(\$ \(\le\) 20\$) 程度であれば良い性能を示している
\cite{noro97-ap,asir-para}.しかし計算の逐次部分,通信コストのために,
性能限界を持つ.

\cite{strategy-accurate}では,
選択戦術を忠実に守りつつ,ペアに関する簡約
(\({{{\mathrm{SPOL}}(g_i,g_j)\!\downarrow_{G}}}\))
を並列に行っている.\$G\$を共有し,複
数のワーカが別々の簡約を行う.以後,この並列化を*ペア並列*と呼ぶ.
ペア並列では,

\begin{itemize}
\item 逐次部分がないが,

\item 中間基底の生成順序を保つため,S多項式の生成,簡約化に待ち が生ずる,

\item 無駄な計算 (0簡約される基底を用いたペア)が生ずる.
\end{itemize}

この方式では, 中間基底の生成順序による待ちがボトルネックとなり,
様々な問題に対して性能限界が生じることが報告されている.
この論文中,斉次な基底計算の場合,生成順序による待ちが大幅に減らせ,
高い並列性能を示すことが言及されているが,その性能は示されていない.

\section{並列算法の組合わせによる並列度の向上}
\label{sec:org6cd1992}

前章の二つの並列化算法はそれぞれ性能限界を持つ.
しかし,その限界を持つ原因は異なるので,二つを組み合わせることにより,
並列性能向上が期待できる.

提案する算法の基本的な考え方は,

\begin{itemize}
\item ペア並列度を検出し,

\item ペア並列度が低い場合に,一簡約並列を行う
\end{itemize}

であるが,ペア並列度の検出は計算中には行えない.そこで,
まず同じ戦術のmodular 計算を行い,
0簡約される基底,基底の生成順序と簡約依存性をあらかじめ求める.
この手法は,\cite{para-gb}で用いられていて,ペア並列度は低いことが
報告されている.つまり,ペア並列度だけでは高い性能向上は見込めない.
そこで,

\begin{itemize}
\item modular 計算により基底の生成順序と簡約依存性をあらかじめ求め,
並列計算可能なブロックに分ける.(これを*並列計算のシナリオ*と呼ぶ)

\item シナリオにより,ブロック内をペア並列実行するが,並列度が投入でき
るプロセッサ台数より小さい場合,全プロセッサが計算に参加できるように,
一簡約並列を併用する.
\end{itemize}

\section{\$\mathbf{d}\$-Gröbner 基底によるペア並列度の向上}
\label{sec:orgebd9c48}

選択戦術として斉次化あるいはsugar を用いる場合には,
あらかじめ決めることができるペア並列度が存在する.

\subsection{\$\mathbf{d}\$-グレブナー基底}
\label{sec:org8ba25ca}

S多項式の全次数(またはsugar次数) \(d\) で打ち切った Buchberger
算法の結果を \(G_d\) とする.この\$G\_d\$のことを*\$d\$-グレブナー基底*
という.

[th-d] 斉次多項式 \(f_1, \ldots, f_n\)
に対する\$d\$-グレブナー基底は以下の性質 を持つ:

\begin{enumerate}
\item \$\textdegree{}(f) < d \$ な \(f\) に対し,
\({{\!\stackrel{G_d}{\longrightarrow^{*}} }}\) が定義さ れる.

\item \(\forall p \in {\mathcal{I}} \ \deg(p) \le d\) \(\Rightarrow\)
\({{p\!\stackrel{G_d}{\longrightarrow^{*}} 0}}\)

\item \(\forall f, g \in G_d\)
\$\textdegree{}(\{\mathrm{HT}\}(f),\{\mathrm{HT}\}(g)) \(\le\) d \$に対し,
\({{{\mathrm{SPOL}}(f, g)\!\stackrel{G_d}{\longrightarrow^{*}} 0}}\)
\end{enumerate}

\(\forall d > d_\infty \ G_d = G_{d_\infty}\) となる
\$d\_\(\infty\)\$が存在する. □

任意の多項式に対し,定理[th-d]の\$\textdegree{}\$を\(\deg_S\) で置き換えて,性 質 1,
2, 3 および\(d_{infty}\) の存在が成り立つ.

定理より\$d\$-グレブナー基底は,

$$G_0 \rightarrow
G_1 \rightarrow
\cdots \rightarrow
G_{d} \rightarrow 
G_{d+1} \rightarrow 
\cdots \rightarrow
G_{d_\infty} = \cdots$$

のように計算でき,\(G_d = G_{d-1} + \{d\mbox{-次式}\}\) となる.

\subsection{\$\mathbf{d}\$-グレブナー基底の並列性}
\label{sec:org484f36f}

前節の定理より,\$G\(_{\text{d-1}}\)\$が求まっていて,\$G\(_{\text{d}}\)\$を求める場合は,
次の事が言える.

\begin{enumerate}
\item \(G_d\) に追加される基底は, \({\mathrm{SPOL}}(g_i, g_j)\),
\(g_i, g_j \in G_{d-1}\), より作られ,基底候補のS多項式に依存性はない.

\item \({{{\mathrm{SPOL}}(g_i, g_j)\!\downarrow_{G_{d-1}}}}\) の計算にも
依存性はない.

\item 上の計算後,\({{{\mathrm{SPOL}}(g_i, g_j)\!\downarrow_{G_{d}}}}\)
の計算は, 1,2 で作られた \$d\$-次基底のみの相互簡約で求められる.
\end{enumerate}

つまり,S多項式の並列生成,\$G\(_{\text{d-1}}\)\$に関する並列簡約,が可能である.
\$d\$-次基底の相互簡約には基底間の依存性が存在するが,
これは一簡約並列実行可能である.

\section{実装と性能(予測)}
\label{sec:orgebe8bf8}

前章により,斉次あるいはsugar を用いた並列算法は,

\begin{itemize}
\item modular 計算によりシナリオを作成し,

\item \$d\$-のS多項式 \$s\_i\$を並列生成し,

\item \({{s_i\!\downarrow_{G_{d-1}}}}\) を並列計算する.ペア並列度が足りな
い場合に,一簡約並列を併用する.

\item \$\{\{s\_i$\backslash$!\(\downarrow_{\text{G}_{\text{d-1}}}\)\}\}\$同士の相互簡約を一簡約並列計算する.
\end{itemize}

となる.asir 上で逐次版の\$d\$-グレブナー基底計算を実装し,その実行過
程を検討し,並列版を現在実装中である.

表[tab-1]に,McKay\cite{noro97-mckay}問題に対し,選択戦術としてsugar戦術をもちいて
実行した結果をしめす.8台の場合の一簡約並列性能は,5.6, ほぼ7割である.
表中の基底数が,シナリオを用いて計算した場合の
ペアの並列度になる.計算時間のもっともかかる,sugar値15,
16辺りのペア並列 度はかなり大きい.sugar値17以上では,
ペアの並列度は1で,ペア並列だけで
は十分な性能向上ははかれないことがわかる.

rrr sugar値\& 基底数\& 時間\\
11 \& 14 \& 21.22\\
12 \& 24 \& 89.32\\
13 \& 37 \& 359.4\\
14 \& 63 \& 2962\\
15 \& 101 \& 84620\\
16 \& 168 \& 572100\\
17 \& 1 \& 28900\\
18 \& 1 \& 12800\\
20 \& 1 \& 30000\\
total \& 442 \& 731800\\

[tab-1]

表[tab-2]に,同じ問題の modular基底を,\$d\$-Gröbner基底算法を用
いて計算した結果を,asir のgr$\backslash$\_mod$\backslash$\_main, F\(_4\) の結果とともに示す.括
弧内は g.c. 時間である.
この計算は並列化のシナリオを作成する部分に相当する. まだ asir F\(_4\)
の性能には及ばないが,gr$\backslash$\_mode$\backslash$\_mainに比べて数割早くなってい
ることがわかる.

rr|rr|rr \& \&\\
180 \& (409) \& 240 \& () \& 126 \& (432)\\

[tab-2]

表[tab-3]に,\$d\$-グレブナー基底計算中の各S多項式の\$G\(_{\text{d-1}}\)\$に関する簡約時間,
\$d\$-次の基底間の相互簡約にかかる時間を示す.
\$G\(_{\text{d-1}}\)\$に関する簡約時間が支配的であり,並列化した場合,ペア並列度
が実行時間に大きく寄与することがわかる.

r|rr|rr|rr \& \& \&\\
total \&180.7 \& (409.3)\& 148.1 \& (321.2)\& 32.2 \& ( 87.3)\\
11\& 0.8 \& ( 3.3)\& 0.7 \& ( 3.1)\& 0.1 \& ( 0.2)\\
12\& 2.5 \& ( 9.6)\& 2.2 \& ( 8.4)\& 0.3 \& ( 1.2)\\
13\& 7.6 \& ( 27.7)\& 6.8 \& ( 24.5)\& 0.8 \& ( 3.2)\\
14\&19.4 \& ( 58.7)\& 16.6 \& ( 49.4)\& 2.7 \& ( 9.1)\\
15\&48.7 \& (134.4)\& 39.6 \& (104.3)\& 9.0 \& ( 29.8)\\
16\&74.6 \& (150.3)\& 54.9 \& (106.2)\& 19.4 \& ( 43.6)\\
17\&25.2 \& ( 22.7)\& 25.2 \& ( 22.7)\& 0.0 \& ( 0.0)\\
18\&1.0 \& ( 0.9)\& 1.0 \& ( 0.9)\& 0.0 \& ( 0.0)\\
19\&0.1 \& ( 0.0)\& 0.1 \& ( 0.0)\& 0.0 \& ( 0.0)\\
20\& 0.3 \& (0.2)\& 0.3 \& (0.2)\& 0.0 \& ( 0.0)\\
21\& 0.4 \& ( 0.3)\& 0.4 \& ( 0.3)\& 0.0 \& ( 0.0)\\

[tab-3]

一簡約並列算法(共有メモリ版)の性能は,12のプロセッサで8程度の並列性能
を得ている\cite{asir-para}.\$d\$-グレブナー基底計算の並列版は実装中で
あるので,算法の組合わせによる全体性能を示すことはできないが,相互簡約
の部分の並列化,ペア並列性の低い部分,が高速化でき,良い性能が得られる
ることは明らかだろう.

para-asir

Attardi, G., Tracerso, C.,: Strategy--Accurate Parallel Buchberger
Algorithms, J.Symb. Comp., 21/4-6 (1997), 411--426

Beker,T., Weispfenning, V.: Gröbner Bases. GTM bf 141, Springer-Verlag,
1993

Faugére, J.C.: Parallelization of Gröbner basis \emph{Proc. PASCO'94}, 1994,
124--132

Faugére, J.C.: A new efficient algorithm for computing Gröbner bases
(\(F_4\)), \emph{Journal of Pure and Applied Algebra} *139*(1--3), 1999, 61--88

Giovini, A., Mora, T., Niesi, G., Robbiano, L., Traverso, C.: "One sugar
cube, please" OR Slection strategies in the Buchberger algorithm, Proc.
ISSAC'91, 1991, 49--54

Noro, M., Kando, T., Takeshima, T.: Solving a large scale problem by
parallel algebraic computation on AP3000, \emph{Research Report ISIS-RR-97},
FUJITSU LABS, 1997

Noro, M., Mckay, J.: Computation of replicable functions on Risa/Asir,
\emph{Proc. PASCO'97}, ACM Press, 1997, 130--138

鈴木正幸: 分散共有メモリを用いた並列Gröbner基底計算の性能評価,
第8回数式処理大会, 1999
\end{document}
