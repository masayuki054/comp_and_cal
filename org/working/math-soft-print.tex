% Created 2018-11-12 月 11:40
% Intended LaTeX compiler: pdflatex
\documentclass[dvipdfmx,11pt]{jarticle}
\usepackage[utf8]{inputenc}
\usepackage[T1]{fontenc}
\usepackage{graphicx}
\usepackage{grffile}
\usepackage{longtable}
\usepackage{wrapfig}
\usepackage{rotating}
\usepackage[normalem]{ulem}
\usepackage{amsmath}
\usepackage{textcomp}
\usepackage{amssymb}
\usepackage{capt-of}
\usepackage{hyperref}
\setlength{\textwidth}{20cm}
\setlength{\oddsidemargin}{-1cm}
\setlength{\evensidemargin}{-1cm}
\setlength{\topmargin}{-3cm}
\setlength{\textheight}{28cm}
\author{鈴木正幸,非常勤講師}
\date{\today}
\title{数学ソフトウェアとSagemath}
\begin{document}

\maketitle

\section{MathLibre}
\label{sec:org8a8d7d4}

\href{https://www.geogebra.org/m/hShSTr6e}{KNOPPIX/Math->MathLibre} 

\begin{itemize}
\item DVD起動Linuxで,

\item オープン・ソースで,

\item フリーな数学ソフトウェアを収録
\end{itemize}

\href{https://www.geogebra.org/m/hShSTr6e}{数学ソフトウェア紹介 - GeoGebraBook}


\section{\href{https://ja.wikipedia.org/wiki/\%E6\%95\%B0\%E5\%AD\%A6\%E3\%82\%BD\%E3\%83\%95\%E3\%83\%88\%E3\%82\%A6\%E3\%82\%A7\%E3\%82\%A2}{数学ソフトウェア - Wikipedia} (1)}
\label{sec:org4a25515}

\begin{enumerate}
\item ソフトウェア電卓  

\href{http://ja.numberempire.com/}{数の帝国 - 数学ツール} 数の帝国は、強力な数学ツールと数について
の知識のコレクションです。

\item 数式処理システム

\begin{itemize}
\item \href{https://ja.wikipedia.org/wiki/Sage\_(\%E6\%95\%B0\%E5\%BC\%8F\%E5\%87\%A6\%E7\%90\%86\%E3\%82\%B7\%E3\%82\%B9\%E3\%83\%86\%E3\%83\%A0)}{Sage (数式処理システム) - Wikipedia}

\item \href{https://www.math.nagoya-u.ac.jp/\~naito/lecture/2012\_SS.calc/software-1.pdf}{数学ソフトウェアの使い方} 名古屋大学医学部 数学通論I 

大学での教養として,mathematica, sage を教えている

\item \href{http://www.math.kobe-u.ac.jp/\~noro/subutsu05.pdf}{数物科学概論/計算代数入門} 神戸大学 野呂さん

だいぶ専門的ですが
\end{itemize}

\item 数値解析

\begin{itemize}
\item \href{https://matome.naver.jp/odai/2136163231573327601}{MATLABの代わりに使えるソフトウェアまとめ- NAVER まとめ}

\item \href{http://signalprocessor.blogspot.jp/2016/}{信号処理のお仕事メモ: 2016} (Octave)

\item \href{http://www.inaba-lab.org/wiki/index.php/Octave\%E5\%85\%A5\%E9\%96\%80}{Octave入門 - 東海大学 コンピュータ応用工学科 稲葉研究室Wiki}
\end{itemize}
\end{enumerate}


\section{\href{http://www.sagemath.org}{Sagemath} とは}
\label{sec:orgdf82f57}

\href{http://www.gregorybard.com/Sage.html}{Gregory V. Bard} 曰く

\begin{itemize}
\item フリーでオープン・ソースで,
\item Maple, Mathematica, Magma, and Matlab に並らぶ,
\item 数学科の学生に最適な
\item 「コンピュータ代数」システム
\end{itemize}

\subsection{クラウド・サーバ -- Sage on the Web}
\label{sec:org14f449b}

\begin{itemize}
\item Webで動く,
\item ノートやデスクトップPCへのインストールは必要ない。
\end{itemize}

\href{http://www.cocalc.com/}{CoCalc.com Sage クラウド・サーバ} 

\begin{itemize}
\item 長めの問題向き,
\item プログラムの保存ができる
\item 登録とログインが必要,
\end{itemize}

\subsection{セル・サーバ -- Sage on the Web}
\label{sec:org969e24a}

\href{http://sagecell.sagemath.org/}{SageMathCell Server}

\begin{itemize}
\item 短かめの問題向き

\item \href{http://www.gregorybard.com/videos/Sage\_part1.swf}{関数,微分,積分,2次元プロット} の例題動画

\item \href{http://www.gregorybard.com/videos/Sage\_part2.swf}{因数分解,3次元プロット}
\end{itemize}

例題

\begin{verbatim}
x,y = var('x y')
plot3d(sin(x+y), [x,-pi,pi], [y, -pi, pi])
\end{verbatim}

\subsection{Sagemath アプリ}
\label{sec:org0988871}

数式処理が,スマホで動くなんて,ほんとにビックリ

\begin{itemize}
\item \href{https://itunes.apple.com/jp/app/sage-math/id496492945?mt=8}{Sage Mathを App Store で}

\item \href{https://play.google.com/store/apps/details?id=org.sagemath.droid\&hl=ja}{Sage Math - Google Play の Android アプリ}
\end{itemize}

\subsection{入門}
\label{sec:org28873eb}

\begin{itemize}
\item \href{http://doc.sagemath.org/html/ja/tutorial/index.html}{Sageチュートリアル.ja} すぐ後で,実行しながら,説明します

\item \href{http://doc.sagemath.org/pdf/ja/tutorial/tutorial-jp.pdf}{Sage tutorial-jp.pdf}

\item \href{http://www.pwv.co.jp/\%7Etake/TakeWiki/index.php?sage\%2Fsage\%E3\%81\%AE\%E7\%B4\%B9\%E4\%BB\%8B}{sage/sageの紹介 - PukiWiki} たけもとさん

\begin{itemize}
\item \href{http://www.pwv.co.jp/\~take/TakeWiki/index.php?sage\%2F\%E8\%A8\%88\%E7\%AE\%97\%E3\%81\%97\%E3\%81\%A6\%E3\%81\%BF\%E3\%82\%88\%E3\%81\%86}{sage/計算してみよう - PukiWiki}
\end{itemize}

\item \href{http://www.sagemath.org/help-video.html}{series of videos/screencasts}
\end{itemize}

\subsection{例題}
\label{sec:org30e8fae}

\begin{itemize}
\item \href{http://doc.sagemath.org/html/en/constructions/index.html}{プログラムの制作例}
\begin{itemize}
\item いろいろな例題と解答
\end{itemize}

\item \href{https://wiki.sagemath.org/interact}{対話的なデモリポジトリ} 
\begin{itemize}
\item interact / calculus / Taylor  を見てみよう
\end{itemize}
\end{itemize}

\subsection{Sagemath に関する本 (フリー)}
\label{sec:org05bb385}

\begin{itemize}
\item \href{http://www.gregorybard.com/Sage.html}{Sage for undergraduates, free pdf}   このページ内に pdf へのリンクが

\item \href{http://mosullivan.sdsu.edu/Teaching/sdsu-sage-tutorial/index.html}{Welcome to the SDSU Sage Tutorial}
\end{itemize}

\subsection{Sagemath に関するいろいろなページ}
\label{sec:org9f9d3af}

\begin{itemize}
\item \href{http://sk.sagepub.com/reference}{SAGE Knowledge - Reference} reference の検索

\item \href{https://qiita.com/HirofumiYashima/items/6bb5770961a3b7d33118}{Sageに関するリンク集}

\item \href{http://wiki.sagemath.org/quickref}{large collection of quick-reference cards}

\item \url{http://sagemath.org} \url{http://doc.sagemath.org}
\end{itemize}


\subsection{Octave との連携}
\label{sec:org02daf40}

\begin{itemize}
\item \href{http://www.pwv.co.jp/\%7Etake/TakeWiki/index.php?sage\%2FSage\%E3\%81\%A7Octave\%E3\%82\%92\%E4\%BD\%BF\%E3\%81\%86}{sage/SageでOctaveを使う - PukiWiki}

\item \href{http://www.inaba-lab.org/wiki/index.php/Octave\%E5\%85\%A5\%E9\%96\%80}{Octave入門 - 東海大学 コンピュータ応用工学科 稲葉研究室Wiki}
\end{itemize}

\subsection{\LaTeX{}とSage}
\label{sec:org255371c}

\begin{itemize}
\item オンライン\LaTeX{}サービス  \href{https://oku.edu.mie-u.ac.jp/\~okumura/texonweb/}{\TeX{} を使ってみよう}

\item \href{https://mytexpert.osdn.jp/index.php?LaTeX\%A4\%CB\%A4\%E8\%A4\%EB\%CF\%C0\%CA\%B8\%BA\%EE\%C0\%AE\%A4\%CE\%BC\%EA\%B0\%FA\%A4\%AD}{\LaTeX{}による論文作成の手引き - MyTeXpert}

\item \href{http://sage.math.gordon.edu/home/pub/51/}{Using \LaTeX{} in Sage -- Sage}
\end{itemize}

\subsection{\LaTeX{} 出力の例}
\label{sec:org584e44c}


\begin{verbatim}
integrate(sin(x^2)/e^(2^x^2),x)
\end{verbatim}
\begin{center}


\begin{verbatim}
integrate(e^(-2^(x^2))*sin(x^2), x)
\end{verbatim}

\begin{verbatim}
latex(integrate(sin(x^2)/e^(2^x^2),x))

\end{verbatim}




\(\int e^{\left(-2^{\left(x^{2}\right)}\right)}\sin\left(x^{2}\right)\,{d x}\)
\end{center}
\end{document}
