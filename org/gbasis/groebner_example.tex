% Created 2017-11-27 月 23:47
\documentclass[a4j]{jarticle}
\usepackage[utf8]{inputenc}
\usepackage[T1]{fontenc}
\usepackage{fixltx2e}
\usepackage{graphicx}
\usepackage{longtable}
\usepackage{float}
\usepackage{wrapfig}
\usepackage{rotating}
\usepackage[normalem]{ulem}
\usepackage{amsmath}
\usepackage{textcomp}
\usepackage{marvosym}
\usepackage{wasysym}
\usepackage{amssymb}
\usepackage{hyperref}
\tolerance=1000
\author{Masayuki Suzuki}
\date{\today}
\title{groebner\_example}
\begin{document}

\maketitle
\tableofcontents

\section{Groebner基底と固有値法}
\label{sec-1}

有理数(QQ) を係数とする多変数($x_1, x_2$) の
多項式環 $QQ[x_1, x_2]$ を生成し,
不定元\$x$_{\text{1}}$, x$_{\text{2}}$\$を取り出す:

\begin{verbatim}
x_1, x_2 = QQ['x_1,x_2'].gens()
\end{verbatim}


\$x$_{\text{1}}$, x$_{\text{2}}$, x$_{\text{3}}$\$を使って多項式を定義する:
\begin{verbatim}
f_1 = 2 * x_1^3 * x_2 +6*x_1^3 - 2* x_1^2 - x_1* x_2 - 3* x_1 - x_2 +3
f_2 = x_1^3*x_2 + 3*x_1^3 + x_1^2*x_2 + 2*x_1^2
f_3 = 3*x_1^2*x_2 + 9*x_1^2 + 2*x_1*x_2 + 5*x_1 + x_2 -3
f_1
f_2
f_3
\end{verbatim}

$f_1$, $f_2$, $f_3$ を基底とするイデアル $I$ を生成する:
\begin{verbatim}
I = ideal(f_1, f_2, f_3)
\end{verbatim}

$I$ のグレブナー基底を求める:

\begin{verbatim}
B = I.groebner_basis(); B
\end{verbatim}

\$QQ[x$_{\text{1}}$, x$_{\text{2]}}$/I\$は,$1, x_2, x_1$ が基底となることがわかる。

\begin{verbatim}
type(B)
\end{verbatim}

\$QQ[x$_{\text{1}}$, x$_{\text{2]}}$/I\$における,かけ算表を作成する:

\begin{verbatim}
bases = [1,x_1,x_2]
x_1p = [I.reduce(x_1*a) for a in bases]; x_1p
x_2p = [I.reduce(x_2*a) for a in bases]; x_2p
\end{verbatim}

\begin{verbatim}
P1 = [ f.coefficient(b) for b in [{x_1:0, x_2:0}, {x_1:1, x_2:0}, {x_1:0, x_2:1}] for f in x_1p]; P1
P2 = [ f.coefficient(b) for b in [{x_1:0, x_2:0}, {x_1:1, x_2:0}, {x_1:0, x_2:1}] for f in x_2p]; P2
\end{verbatim}


\begin{verbatim}
M1 = matrix(QQ,3,3,P1); M1
M2 = matrix(QQ,3,3,P2); M2
\end{verbatim}

\begin{verbatim}
s1 = M1.eigenvalues(); s1
s2 = M2.eigenvalues(); s2
\end{verbatim}

\begin{verbatim}
[[i,j,yf_1.subs({x_1:s1[i], x_2:s2[j]}), f_2.subs({x_1:s1[i], x_2:s2[j]})] 
  for i in range(3) for j in range(3)]
\end{verbatim}
% Emacs 25.1.1 (Org mode 8.2.10)
\end{document}
