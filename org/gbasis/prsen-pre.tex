\documentclass[dvipdfmx,11pt,notheorems]{beamer}
%%%% 和文用 %%%%%
\usepackage{bxdpx-beamer}
\usepackage{pxjahyper}
\usepackage{minijs}%和文用
\renewcommand{\kanjifamilydefault}{\gtdefault}%和文用

%%%% スライドの見た目 %%%%%
\usetheme{Madrid}
\usefonttheme{professionalfonts}
\setbeamertemplate{frametitle}[default][center]
\setbeamertemplate{navigation symbols}{}
\setbeamercovered{transparent}%好みに応じてどうぞ)
\setbeamertemplate{footline}[page number]
\setbeamerfont{footline}{size=\normalsize,series=\bfseries}
\setbeamercolor{footline}{fg=black,bg=black}
%%%%

%%%% 定義環境 %%%%%
\usepackage{amsmath,amssymb}
\usepackage{amsthm}
\theoremstyle{definition}
\newtheorem{theorem}{定理}
\newtheorem{definition}{定義}
\newtheorem{proposition}{命題}
\newtheorem{lemma}{補題}
\newtheorem{corollary}{系}
\newtheorem{conjecture}{予想}
\newtheorem*{remark}{Remark}
\renewcommand{\proofname}{}
%%%%%%%%%

%%%%% フォント基本設定 %%%%%
\usepackage[T1]{fontenc}%8bit フォント
\usepackage{textcomp}%欧文フォントの追加
\usepackage[utf8]{inputenc}%文字コードをUTF-8
\usepackage{otf}%otfパッケージ
\usepackage{lxfonts}%数式・英文ローマン体を Lxfont にする
\usepackage{bm}%数式太字
%%%%%%%%%%
%% \title[略タイトル]{グレブナー基底と高次多変数方程式の解法}%[略タイトル]{タイトル}
\author[まさゆき]{鈴木 正幸}%[略名前]{名前}
\institute{岩大・非常勤講師}%[略所属]{所属}
\date{\today}%日付
\begin{document}

\begin{frame}[plain]\frametitle{}
\titlepage %表紙


\begin{center}
{\large 
変数が多く,\\
次数が高い,\\
方程式の解を, \\
求めるアルゴリズム}
\end{center}
\end{frame}

%% \begin{frame}\frametitle{Contents}
%%   \tableofcontents %目次
%% \end{frame}


\section{方程式を解くとは?} 
\begin{frame}\frametitle{一次方程式}

 \[a x = b \] 
  
\begin{center}
両辺に,$a^{-1}$ を,左から掛ける:
\end{center}

 \[ x = a^{-1} b \]

\end{frame}

\begin{frame}\frametitle{連立一次方程式 (系)}
\[
    \begin{array}{rcl}
      a_{11} x_1 + a_{12} x_2 + \ldots & = & b_1,\\
      &  \cdots &  \\
      a_{n1} x_1 + a_{n2} x_2 + \ldots & = & b_n
    \end{array}
    \]
\begin{itemize}
	\item 線形代数, ガウスの消去法
	\item 一次方程式, $a x = b $ に帰着させる
\end{itemize}
\end{frame}

\begin{frame}\frametitle{一変数方程式}
  \[
  a_n x^n + \cdots + a_0 = b
  \]
  
  \begin{itemize}
  \item 解の公式, $x^n = c $ に帰着させる.
  \item 帰着できない時, 数値計算 (ニュートン法) で近似的に求める.
  \end{itemize}
\end{frame}

\begin{frame}\frametitle{多変数(代数)方程式(系)}
  \[
  \begin{array}{rcl}
    f_1(x, y, \ldots, z) & = & 0, \\
    &\cdots&, \\
    f_n(x, y, \ldots, z) & = & 0
  \end{array}
  \]
  
\begin{itemize}
\item 変数消去, 因数分解
\item 必ず解ける方法を知っていますか?
\end{itemize}
\end{frame}

\subsection{数の集合と基底と最大公約数}

\begin{frame}\frametitle{天秤秤の問題}
\begin{quote}
天秤秤と, $a$グラムの重りと $b$グラムの重りが無数にあるとします.
どんな重さが測れるでしょう?

\ \\

あるいは, $c$ グラムを測る事ができますか?
\end{quote}
\end{frame}

\begin{frame}\frametitle{数の集合と基底}

\begin{block}{$ P = \left\{ a x + b y | x, y \in {\cal Z} \right\} $}

$a$の倍数と$b$の倍数を加えてできる整数の集合$P$を考えます。

$a$ と $b$ は,$P$ を生成する基底です。

\end{block}

$ c \in P $ ならば問題は解決です。

\begin{block}{$ \left\{ g z | z \in {\cal Z}, g \in P \right\} = P$ となる,$g$ があるか?}

\begin{itemize}
\item $a$ と $b$ の組合わせで作れる最小の数は,最大公約数 $g$ であ
り,それの倍数しか$a$ と $b$ の組合わせでは作れません.

\item $ a x + b y = g $となるの$x$と$y$は,ユークリッドの互除法によって求められます.

\item 最大公約数$g$は, $a$ と $b$ の組合わせでできる数の集合の
{\bf もっとも簡単な基底} となります.
\end{itemize}
\end{block}
\end{frame}

\begin{frame}\frametitle{
二つの方程式  $f_1(x)=0, f_2(x)=0$ の共通解は?}

  それぞれの方程式の解を求めて, 共通な解を求めてもいいですが,

  前の議論から,

\begin{itemize}
\item 二つの式 $f_1(x), f_2(x)$ の組み合わせで
  できる, 多項式全ての集合を考える:

    \[
\left\{A(x) f_1(x) + B(x) f_2(x) | A(x), B(x) \mbox{は任意の$x$の多項式}\right\}
\]


\item 最も簡単な(次数の低い)式 (基底) を求め,
\item その解を求める.
\end{itemize}  
\end{frame}

\begin{frame}\frametitle{最大公約多項式}
\begin{itemize}
\item この基底は,$f_1(x)$と$f_2(x)$の最大公約多項式($g(x)$)となり,
  \[
  \begin{array}{lcr}
    a(x) f_1(x) + b(x) f_2(x) & = & g(x),\\
    \deg(a(x)) < \deg(f_2(x)),\\
    \deg(b(x)) < \deg(f_1(x)) 
  \end{array}
  \]

\item $g(x)$, $a(x)$, $b(x)$ はユークリッドの互助法で求められる
  
\end{itemize}
\end{frame}

\subsection{多変数方程式をどう解くか?}

\begin{frame}\frametitle{多変数で高次な方程式をどう解くか?}
\[
  \begin{array}{rcl}
    f_1(x, y, \ldots, z) & = & 0, \\
    &\cdots&, \\
    f_n(x, y, \ldots, z) & = & 0
  \end{array}
\]

$f_1, \ldots, f_n$ を組合わせでできる任意の多項式の集合を考える:

\[ 
\left\{ A_1(x, \ldots, z) f_1(x, \ldots, z) + \cdots + A_n(x, \ldots, z)
 f_n(x, \ldots, z) \right\}
\]

この集合を$(f_1, ... , f_n)$と表し,$f_1$から$f_n$ が作る
{\bf イデアル}${\cal I}$と呼ぶ.

${\cal I}$を作ることのできる多項式の組をイデアルの基底と呼びます.

\end{frame}

\begin{frame}\frametitle{都合の良い基底}

方程式を解くのに都合の良い基底を求めることは,

\ \\

同じ解を持つ,より簡単な方程式系への変換となる.
この基底が例えば,

\[
(g_1(x, z) =0, g_2(y,z) = 0, \cdots, g_m(z) =0)
\]

という形で求まれば,

多変数方程式の問題は, 一変数方程式の
問題に帰着される.

\begin{itemize}
\item 「このような変形はできるのか」,
\item 「変形する方針は」,
\item 「必ず求まるのか」
\end{itemize}

などが問題となる.
\end{frame}

\section{パズルと基底と簡約}

\begin{frame}\frametitle{グラス置き換えパズル}

ウィスキーのグラス $W$, ビールのグラス $B$, お酒のグラス $S$ が
一列に並んでいる.

グラスは次の置き換え規則で, 置き換えて良いとする.

\[
\mbox{置き換え規則} G \left\{ 
 \begin{array}{rll}
B &  \leftarrow\rightarrow  &  W B\\
BS& \leftarrow\rightarrow   & W  \\
 \end{array}
\right.
\]

\begin{block}{問題}
\begin{enumerate}
\item $BSBS$ は $WWWB$ に置き換えできるか?
\item $BSBBS$ は $BWW$ に置き換えできるか?
\end{enumerate}
\end{block}
\end{frame}

\begin{frame}\frametitle{問題の難しい点}
\begin{itemize}
\item できる場合はその置き換えを示せば良いが,
\item できない事を示す事.
\end{itemize}
\end{frame}

\begin{frame}\frametitle{パズル解法への道}
\begin{block}{簡単な方へ置き換える (簡約化)ことにする}
\[
\mbox{簡約規則} R \left\{ \begin{array}{rll}
WB & \rightarrow  &  B\\
BS & \rightarrow   & W  \\
	  \end{array}
\right.
\]
\end{block}

\begin{block}{正規形}
\begin{itemize}
\item これ以上簡約できないものを{\bf 正規形}と言う

\item 置き換え規則 $G$ で置き換え可能な列の要素は
      簡約規則 $R$ で同じ正規系を持つか?

\item この性質が成り立てば, 簡約系で正規形が同じであれば, 
      置き換え系で, 置き換え可能となる.
\end{itemize}
\end{block}
\end{frame}

\begin{frame}\frametitle{簡約規則の追加}

置き換え可能なのに, 同じ正規形を持たない場合は,
そのような簡約規則を追加すればよい.

例えば, $WBS$ は二つの置き換えが可能:

\[
\left\{ \begin{array}{rllll}
WBS & \rightarrow  &  WW\\
WBS & \rightarrow   & BS & \rightarrow  & W\\
	  \end{array}
\right.
\]

置き換え系では, $WW$ と $W$ は, $WBS$ を通して置き換え可能である
から, 簡約系で

\[
WW  \rightarrow   W
\]

を新しい簡約規則として採用すればいい事になる.

この追加される簡約規則をどうやって見付けるかが問題となる.
\end{frame}

\begin{frame}\frametitle{新しい規則を見つける}
\begin{itemize}
\item 簡約規則の左項中で, 重なりが生ずるような二つの規則を探す.

(この二つの簡約規則を{\bf 危険対}と呼ぶ).

今の場合, $BS$ と $WB$ は 重なりを持つ項, $WBS$ を別の正規形に簡
約する可能性を持つ.

\item この操作を次々に繰り返し, 危険対が全て同じ簡約形を持つよう
になった時, 置き換え可能である物は, 全て同じ正規形を持つ事になる.
\end{itemize}

これを,簡約系の{\bf 完備化}という.
\end{frame}

\begin{frame}\frametitle{完備な系}
\begin{itemize}
\item 正規形は有限ステップで求まる. 
({\bf 停止性})
\item ある項の正規形は, 簡約順序によらず同じになる.({\bf 合流性})
\end{itemize}
\end{frame}

\begin{frame}\frametitle{パズルの答え}

簡約規則 $R$ を完備化すると:

\[
\mbox{簡約規則} R' \left\{ \begin{array}{rll}
WB & \rightarrow  &  B\\
BS & \rightarrow   & W  \\
WW & \rightarrow   & W  \\
	  \end{array}
\right.
\]

\begin{block}{これでパズルの問題が解ける:}
\begin{itemize}
\item $BSBS \rightarrow^* W$, 
$WWWB \rightarrow^* B$, なので, 置き換え不可

\item $BSBBS \rightarrow^* BW$, 
$BWWW \rightarrow^* BW$, なので, 置き換え可
\end{itemize} 
\end{block}
\ \\
\begin{alertblock}{これがどう方程式と関係しているのでしょう?}
\end{alertblock}
\end{frame}

\section{グレブナー基底}
\begin{frame}\frametitle{グレブナー基底}

与えられた方程式$f_i$ の最高順位項を $head(f_i)$ 、残りの項を 
$rest(f_i)$ とすると, 
\[ f_i = head(g_i) + rest(g_i)  = 0 \]
から
\[ head(g_i) \rightarrow - rest(g_i) \]
という簡約規則を作る事ができる.

このような簡約系を作るには, 項間の順序, 簡約, 危険対の求め方を,
方程式用に決める必要がある.
\end{frame}

\subsection{項の間の順序と簡約}
\begin{frame}\frametitle{項の間の順序と簡約}

いくつの順序が考えられ, 順序によって完備な簡約系が異る.

\begin{description}
\item [辞書式順序:]:   $xyz > yz^3 > z^5$ 
\item [全次数辞書式順序:]  $x^5 > x^4y > x^3yz$ 
\end{description}
\end{frame}

\begin{frame}{簡約}
\begin{block}{簡約}
 
基底の先頭項を残りの項で置き換える簡約規則と見て,
項をより低順位項で置き換える操作.
\end{block}

\begin{exampleblock}{$g_1$ を $g_2$ で簡約}
\begin{itemize}

\item $g_1 = x^4yz - xyz^2~~~(~head(g_1) = x^4yz~~,~~rest(g_1) = xyz^2~)~ $

\item $g_2 = x^3yz - xz^2~~~(~head(g_2) = x^3yz~~,~~rest(g_2) = xz^2)~ $

\[
\begin{array}{ll}
g' & = g_1 - ( head(g_1) / head(g_2) ) g_2 \\
   & = g_1 - ( x^4yz / x^3yz ) g_2 \\
   & = x^2z^2 - xyz^2
\end{array}
\]
\end{itemize}
\end{exampleblock}
\end{frame}

\begin{frame}
\begin{block}{S多項式}

新たな簡約規則を得るための計算.

2つの多項式$f_1,f_2$のS多項式を$Sp(f_1,f_2)$と書き、以下のように計算する。
\[
Sp(f_1,f_2)= \frac{lcm}{head(f_1)} f_1 - \frac{lcm}{head(f_2)}f_2 
\]
\end{block}
%%\end{frame}

%%\begin{frame}
\begin{exampleblock}{}
 \[
 \begin{array}{lcrr}
 g_1 & = & x^3yz - xz^2, & head(g_1) = x^3yz,\\
 g_2 & = & x^2y^2 - z^2, & head(g_2) = x^2y^2
 \end{array}
 \]
 \[
 lcm(head(g_1), head(g_2)) = x^3y^2z
 \]
 \[
 \begin{array}{lcr} 
  Sp(g_1,g_2) & = & ( lcm / head(g_1) ) g_1 - ( lcm / head(g_1)) g_2 \\
  & = & ( x^3y^2z / x^3yz ) g_1 - ( x^3y^2z / x^2y^2 ) g_2 \\
  & = & -xyz^2 + xz^3
 \end{array}
\]
\end{exampleblock}
\end{frame}

\begin{frame}\frametitle{グレブナー基底の定義}

イデアル${\cal I}$の基底を$ G = \{ f_1,\cdots,f_n \} $とする。

$F$を可能な限りM簡約した結果を$F'$とし,

\[F \stackrel{G}{\longmapsto} F' \] と表す.

\begin{block}{グレブナー基底$ G $l}

${\cal I}$の任意の要素 $f$に対し,

\[
	f \stackrel{G}{\longmapsto} 0 
\]
\end{block}
\ \\
$ G $がグレブナー基底の時,$f \stackrel{\psi}{\longmapsto} f' $を計算
し,$f'=0$を調べることで、$f \in {\cal I}$であるかを簡単に決定できる.
\end{frame}

\begin{frame}
\begin{exampleblock}{$f_1,f_2,f_3$ のグレブナー基底計算(全次数辞書式順序)}

\[ \left\{
\begin{array}{l}
f_1 = 2{x_1}^3x_2 +6{x_1}^3-2{x_1}^2-x_1x_2-3x_1-x_2+3\vspace{.2in}\\
f_2 = {x_1}^3x_2 + 3{x_1}^3 + {x_1}^2x_2 + 2{x_1}^2\vspace{.2in}\\
f_3 = 3{x_1}^2x_2 + 9{x_1}^2 + 2x_1x_2 + 5x_1 + x_2 -3
\end{array}
\right. \]
\end{exampleblock}
%%\end{frame}

%%\begin{frame}
\begin{exampleblock}{s-多項式}
\[
\begin{array}{ll} 
Sp(f_1,f_2) &= ( lcm / head(f_1) ) f_1 - ( lcm / head(f_1)) f_2 \\
            &= ( 2{x_1}^3x_2 / 2{x_1}^3x_2 ) f_1 - ( 2{x_1}^3x_2 / {x_1}^3x_2 ) f_2 \\
	    &= -2x_1^2 x_2 -6x_1^2 -x_1 x_2 - 3x_1 -x_2 +3 \\
	    &= f'_4
\end{array}
\]
\end{exampleblock}

\begin{exampleblock}{簡約}

\[
\begin{array}{lll}
f'_4 & \stackrel{f_3}{\longmapsto} & f'_4 - (-2x_1^2 x_2 /{head(f_3)})f_3 \\
     & = & x_1 x_2+ x_1 -x_2 +3
\end{array}
\]
\end{exampleblock}
\end{frame}

\begin{frame}
\begin{exampleblock}{グレブナー基底}
$f_1,f_2,f_3$ のグレブナー基底
\[ 
	G = \{
 \begin{array}{l}
  x_1 x_2 + x_1- x_2 + 3, \\
  2{x_1}^2 - 3 x_1 + 2 x_2 - 6, \\
  2{x_2}^2 - 8 x_1 - 5 x_2 -3 \}
 \end{array} 
\]
\end{exampleblock}
\end{frame}

\section{グレブナー基底から方程式の解を求める方法}
\begin{frame}\frametitle{グレブナー基底から方程式の解を求める方法}
辞書式順序で基底計算を行うと、連立方程式の解が求めやすいが、
基底計算に時間がかかる上に計算量が多くなる.

簡単に求まる基底から,解を求める手法として固有値法がある.

\begin{block}{固有値法}
\begin{enumerate}
\item 任意の多項式を, グレブナー基底 $G$ で簡約した多項式の集合 
${\cal P}^s /{\cal I}$ は, ベクトル空間をなす.

\item グレブナー基底の最高順位項で割り切れない全ての項の集合を
Normal setといい、 ${\cal P}^s /{\cal I}$ ベクトル空間の基底となる。

\item Normal set により $x_i \times $ を行列で表す事ができる.

\item その行列の固有値は, ${\cal I}$ の$x_i$ に関する解となる.
\end{enumerate}
\end{block}
\end{frame}

\begin{frame}
\begin{block}{$f_1,f_2,f_3$ のグレブナー基底}
\[ G = [x_1 x_2 + x_1 - x_2 + 3, 2{x_1}^2-3x_1+2x_2-6,2{x_2}^2-8x_1-5x_2-3] \]

\[Normal \; Set =  \{1,x_2,x_1\}\]
\end{block}

\begin{block}{書き換え規則}

\[ \left\{
\begin{array}{rl}
x_1x_2 & \rightarrow  -x_1+x_2-3\vspace{.15in}\\
{x_1}^2 & \rightarrow  \frac{3}{2}x_1-x_2+3\vspace{.15in}\\
{x_2}^2 & \rightarrow  4x_1+\frac{5}{2}x_2+\frac{3}{2}
\end{array}
\right. \]
\end{block}
\end{frame}

\begin{frame}
$P=c_1\vec{x_1} + c_2\vec{x_2} + c_3$ 

\begin{block}{$x_1 \times $ の行列(かけ算表)}

\[
\bordermatrix{
 &  1 & x_2 & x_1\cr
 x_1\times 1   &  0 &  0 &  1 \cr
 x_1\times x_2 & -3 &  1 & -1 \cr
 x_1\times x_1 &  3 & -1 & 3/2 \cr
}
\]
\end{block}

\begin{block}{$x_1 \times$ の固有値}

\[ \left[ 0, \; \frac{5}{4}+\frac{1}{4}\sqrt{65},
\; \frac{5}{4}-\frac{1}{4}\sqrt{65}\right] \]

\end{block}

\end{frame}

\begin{frame}
\begin{block}{$x_2 \times $ の行列(かけ算表)}

\[
\bordermatrix{
 &  1 & x_2 & x_1\cr
 x_2\times 1   &  0  &  1  &  0 \cr
 x_2\times x_2 & 3/2 & 5/2 &  4 \cr
 x_2\times x_1 & -3  &  1  & -1 \cr
}
\]
\end{block}
\begin{block}{$x_2 \times$ の固有値}

\[ \left[ 3, \; -\frac{3}{4}+\frac{1}{4}\sqrt{65},
\; -\frac{3}{4}-\frac{1}{4}\sqrt{65}\right] \]


\end{block}

これらの固有値が $f_1,f_2,f_3$ の解である。
\end{frame}

\end{document}




 page番号が狂うので

